\clearpage\chapter*{Podsumowanie}
    \phantomsection\addcontentsline{toc}{chapter}{Podsumowanie} % Dodanie go do spisu treści
        \noindent Na bazie przeprowadzonych testów stwierdzam, że założony cel pracy został osiągnięty. Powstałe urządzenie spełnia wszystkie założone wymagania techniczne, czyli umożliwia
        bezprzewodowy odczyt poziomu zapełnienia zbiornika asenizacyjnego oddalonego do 180 metrów, z~zadaną dokładnością. Jest w~stanie pracować bez przerwy przez ponad 5 lat bez wymiany baterii, a~jego 
        koszt oscyluje w~granicach od 400 złotych przy produkcji jednostkowej do około 270 złotych przy zamówieniu tysiąca sztuk.
        (Koszt nie uwzględnia ceny wyprodukowania i~montażu płytek PCB).

        W~trakcie testów gotowego urządzenia zauważyłem problem związany z~czujnikiem, który polegał na przekłamywaniu odczytów w~zależności od umiejscowienia czujnika.
        Przyczyną był prawdopodobnie wielokrotnie odbijający się sygnał ultradźwiękowy, co można rozwiązać prostymi algorytmami przetwarzającymi dane. W~najprostszym wypadku można zapamiętywać najmniejsze 
        wskazania czujnika, albo korzystać ze średniej kroczącej odczytów. Uważam, że pomimo spełnienia wymagań jest jeden kluczowy aspekt, który wymaga poprawy i~dalszych badań -- jest to zasięg komunikacji.
        Przy konstruowaniu skupiłem się na optymalizacji kosztów, dlatego też użyłem najtańszych dostępnych komponentów. Wybór anteny kierunkowej o~lepszych parametrach mógłby znacznie wpływnąć na poprawę
        osiągów. Można przewidzieć róże warianty urządzenia i~założyć, że najtańsza wersja korzystałaby z~obecnej anteny, a~wersja lepsza z~anteny kierunkowej -- są to różne potencjalne strategie biznesowe, które
        można wykorzystać.
        
        W~przyszłości konieczne byłoby zapewnienie aktualizacji oprogramowania bezprzewodowo w~ramach FUOTA (ang. \textit{firmware update over the air}), co w~oczywisty sposób podniosłoby komfort użytkowania.
        W~przypadku systemu składającego się z~kilkunastu, czy nawet kilkudziesięciu tego typu urządzeń, ręczna aktualizacja oprogramowania jest niedopuszczalna. Z~punktu widzenia bezpieczeństwa modułu 
        pomiarowego należałoby zastanowić się także nad wykorzystaną obudową. Obecna formuła, czyli druk 3D, może sprawdzić się w~przypadku stacji bazowej, jednak moduł z~czujnikiem wymaga obudowy, która będzie
        w~najwyższych klasach szczelności IP, ponieważ otoczenie, w~jakim pracuje sensor, jest potencjalnie niebezpieczne z~powodu występowania w~nim metanu \cite{Biofos}, który jest gazem łatwopalnym i~w
        stężeniach już od 5\% do 15\% może eksplodować. Stąd też na dalszych etapach rozwoju projektu powinna zostać rozważona możliwość zmiany obudowy. Możliwe jest również zalanie wnętrza obudowy żywicą 
        -- również celem zabezpieczenia samego urządzenia przed kopiowaniem rozwiązań. 

        Uważam, że urządzenie jest warte uwagi, ponieważ na rynku nie ma zbyt wielu podobnych rozwiązań, a~na pewno nie ma takich, które spełniałyby postawione wymagania. Wynika to głównie z~faktu, że 
        większość z~nich opiera się o~przewodowy interfejs między czujnikiem a~odbiornikiem, co jest mało wygodne. Dalszy rozwój obecnego projektu mógłby wiele zmienić. Kolejne wersje mogłby korzystać 
        z~sieci LoRaWAN, co pozwoliłoby teoretycznie na pomiary w~każdym miejscu, do którego dochodziłaby sieć. Nieustannie rozwijający się rynek IoT wraz z~technologiami LPWAN (ang. \textit{low-power wide-area network}),
        do których należy również LoRaWAN, doprowadzi do masowej popularyzacji podobnych czujników, które nie muszą być obecne jedynie w~przemyśle. Tanie czujniki bezprzewodowe dalekiego zasięgu to coś, co może
        przynieść wiele pożytku. W~przyszłości możliwe byłoby rozszerzenie funkcjonalności urządzenia o~pomiary dodatkowych parametrów takich jak temperatura, wilgotność lub obecność szkodliwych gazów, co zwiększyłoby jego atrakcyjność 
        dla potencjalnego użytkownika, a~także uniwersalność. Integracja z~aplikacją mobilną lub chmurową zapewniłaby maksymalny poziom komfortu i~urządzenie nie odstawałoby w~ten sposób od współczesnych 
        czujników oferowanych przez duże firmy. Projekt stanowi solidną podstawę do dalszego rozwoju systemów zdalnego monitorowania, które mogą znaleźć zastosowanie między innymi także w~rolnictwie. 
