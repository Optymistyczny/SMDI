\clearpage\chapter*{Wprowadzenie}
\phantomsection\addcontentsline{toc}{chapter}{Wprowadzenie}
    \noindent Rozwój Internetu Rzeczy (IoT) jest jednym z~głównych filarów Czwartej Rewolucji Przemysłowej. Z~tego powodu inteligentne urządzenia elektroniczne zyskują coraz
    większą popularność, stopniowo dominując rynek konsumencki. Powstaje coraz więcej rozwiązań, których celem jest oszczędzanie czasu oraz
    ułatwianie użytkownikom codziennych czynności. Są to przykładowo obsługiwane zdalnie, z~poziomu telefonu, czujniki temperatury, inteligentne programatory świateł
    lub gniazdek, systemy monitorowania jakości wody\cite{s24092871}, czy systemy zdalnej kontroli wizualnej. 
    Rozwiązania niektórych problemów pozostawiają wciąż szerokie możliwości poprawy. Jednym z~nich jest monitorowanie poziomu zapełnienia przydomowego zbiornika asenizacyjnego.
    Na terenach, do których nie została doprowadzona sieć kanalizacyjna, powszechne jest instalowanie podziemnych, bezodpływowych
    zbiorników na ścieki (w USA szacuje się ich obecność w około 23\% gospodarstw domowych\cite{Solomon2006_SepticRT}). Rozwiązanie to stawia przed użytkownikiem szereg dylematów, począwszy od dobrania odpowiedniego umiejscowienia, poprzez kontrolowanie
    szczelności pojemnika i~stopnia jego zapełnienia, aż do zaplanowania wywozu nieczystości. Następstwa związane z~użytkowaniem nieszczelnego lub przepełnionego
    szamba są wielopoziomowe: bakteriologiczne, toksykologiczne i~prawne. Konsekwencje natury prawnej wiążą się z~karami pieniężnymi, w~kwotach nawet do 500
    złotych. W~świetle obecnie rosnącego znaczenia ekologii, warto również zwrócić uwagę na problem, jakim jest przedostanie się do gruntu materiałów
    zakaźnych, takich jak wirusy, bakterie i~pasożyty oraz toksycznych związków chemicznych. Zatrucie agresywną chemią może być trwałe i~utrzymywać 
    się latami. Konsekwencje są tym groźniejsze, im bliżej uszkodzonego zbiornika znajdują się cieki wodne czy studnie.

    Analiza rynku konsumenckiego wykazała, że dostępne są gotowe rozwiązania opisanego problemu monitorowania poziomu zapełnienia przydomowego szamba.
    Tradycyjne podejścia najczęściej opierają się na rozwiązaniach takich jak pływaki mechaniczne czy czujniki zanurzeniowe, 
    i~są obarczone szeregiem ograniczeń. Wspomniane urządzenia pomiarowe wymagają ciągłego kontaktu z~medium w~agresywnym środowisku, co prowadzi do ich stopniowego
    zużycia i~potencjalnych awarii, a~tym samym kosztownych dla użytkownika częstych przeglądów technicznych i~wymian. Problem z~tradycyjnymi metodami pojawia się
    również podczas regularnego opróżniania zbiornika, kiedy konieczne jest wcześniejsze wyjęcie czujnika lub zastosowanie specjalnych środków ostrożności w~celu
    jego ochrony.
    Wprowadzenie bardziej zaawansowanych, bezkontaktowych metod pomiaru pozwoliłoby na poprawę komfortu użytkowników oraz na uproszczenie monitorowania przydomowego
    szamba.W~wielu branżach coraz częściej stawia się na bezprzewodowe i~energooszczędne urządzenia elektroniczne. Stosowanie
    takich rozwiązań pozwala nie tylko na automatyzację procesu, ale również na integrację systemów z~innymi urządzeniami w~ramach Internetu Rzeczy (IoT -- ang.\ \textit{Internet of Things}).
    Opracowywane są również urządzenia IOT do minotorowania stężenia niebezpiecznych gazów w szambach\cite{Tuyishimire2023_IoTSeptic}. Przegląd dostępnych rozwiązań pozwolił wysnuć wniosek, że większość istniejących urządzeń do pomiaru poziomu zapełnienia zbiornika asenizacyjnego, choć skuteczna,
    nie spełnia wszystkich wymagań gospodarstw domowych. Wady, jakimi są obarczone, to najczęściej wysokie zużycie energii lub konieczność zasilania sieciowego, co
    znacząco ogranicza miejsca, gdzie możemy je zastosować.

    Celem niniejszej pracy jest opracowanie energooszczędnego, bezprzewodowego urządzenia badającego poziom zapełnienia przydomowego zbiornika asenizacyjnego.
    Proponowane rozwiązanie wykorzystuje technologię ultradźwiękową, która, jak pokazano w~następnym rozdziale pracy, gwarantuje wysoką dokładność, bez konieczności kontaktu 
    czujnika z~medium, jakim są ścieki kanalizacyjne. Wykorzystanie zasilania bateryjnego umożliwia stosowanie urządzenia w~każdym miejscu, bez potrzeby stałego 
    podłączenia do sieci elektrycznej. Ważnym aspektem projektu jest również modularność, czyli możliwość dostosowania
    urządzenia do potrzeb użytkownika poprzez zastosowanie innego rodzaju czujników poziomu. Dzięki modularności możliwa jest także łatwa wymiana poszczególnych
    elementów systemu, co pozytywnie wpływa na obniżenie kosztów serwisowania urządzenia i~wydłużenie jego żywotności. Projekt zakłada poszukiwanie równowagi między jakością a~ceną.
    Niniejsza praca ma na celu nie tylko do opracowania funkcjonalnego
    urządzenia, ale też do wykazania, że wbrew temu co sugerują dostępne na rynku urządzenia, możliwe jest połączenie takich cech jak energooszczędność,
    przystępna cena, niezawodność oraz modularność w~jednym. Tym sposobem powstanie system, który nie tylko ułatwi obsługę szamb, ale także
    wpisze się w~globalny trend automatyzacji, optymalizacji procesów monitorowania środowiska oraz rozwoju Internetu Rzeczy.

\cleardoublepage
\part{Analiza stanu wiedzy i dobór narzędzi}
\clearpage\chapter{Przegląd technik pomiarowych i istniejących rozwiązań}
    \noindent Metody pomiaru poziomu cieczy można podzielić między innymi na  bezpośrednie i~pośrednie \cite{Piotrowski2017}.
    Metody bezpośrednie wymagają obecności osoby, która może odczytać wartość pomiaru --  odczyt jest wzrokowy. Metody pośrednie wykorzystują najczęściej jakieś urządzenie
    lub wielkość fizyczną, za~pomocą których pośrednio można zmierzyć poziom cieczy. 
    
    Poziomierze do pomiaru \textbf{bezpośredniego}:
    \begin{itemize}
        \item wziernikowe,
        \item prętowe,
        \item sondujące:
        \begin{itemize}
            \item strunowe,
            \item taśmowe.
        \end{itemize}
    \end{itemize}

    Poziomierze do pomiaru \textbf{pośredniego}:
    \begin{itemize}
        \item siłowe:
            \begin{itemize}
                \item pływakowe,
                \item wyporowe,
                \item wagowe,
            \end{itemize}
        \item ciśnieniowe:
            \begin{itemize}
                \item hydrostatyczne,
                \item piezometryczne,
            \end{itemize}
        \item elektryczne:
            \begin{itemize}
                \item oporowe,
                \item pojemnościowe,
            \end{itemize}
        \item wykorzystujące fale mechaniczne:
            \begin{itemize}
                \item akustyczne,
                \item ultradźwiękowe,
            \end{itemize}
        \item wykorzystujące fale elektromagnetyczne:
            \begin{itemize}
                \item radarowe,
                \item optyczne,
                \item jądrowe.
            \end{itemize}
    \end{itemize}

    W~ramach niniejszej pracy zdecydowano się wykorzystać metody pozwalające na autonomiczne działanie urządzenia. W~związku z~tym metody bezpośrednie w~dalszej części przeglądu
    literaturowego zostaną pominięte. 
        \section{Poziomierze siłowe} 
            \noindent\textbf{Poziomierz pływakowy} opiera swoje działanie na unoszącym się na powierzchni 
            elemencie -- pływaku, którego ruch, na przykład poprzez linę, przekłada się na przemieszczenie ciężarka lub innego przedmiotu będącego poza zbiornikiem, co pozwala uzyskać informację o~szukanym
            poziomie cieczy --~zgodnie ze schematem na rys. \ref{fig:poz_pływak}. Do ich zalet należą: prosta budowa, bardzo duże zakresy pomiarowe (do 40 metrów) i~dokładność \cite{Piotrowski2017}. Do wad należy kontakt pływaka
            z~badanym medium, co może doprowadzić z~czasem do jego degradacji. 
            \begin{figure}[!htbp]
                \centering \includegraphics[width=0.35\textwidth]{poz_pływak.jpg}
                \caption{Schemat poziomierza pływakowego \cite[241]{Piotrowski2017}}
                \label{fig:poz_pływak}
            \end{figure}

            \textbf{Poziomierz wyporowy} jest odmianą opisanego wcześniej poziomierza pływakowego. W~tym wariancie na pływak 1 (rys. \ref{fig:poz_wypor}),
            działa pewna siła wyporu związana z~zanurzeniem, która jest następnie przekazywana za pomocą dźwigni 5 lub sprężyny 2~do czujnika 4 (na przykład tensometrycznego)
            i~przeliczna na szukaną wielkość -- poziom cieczy. Podstawa sprężyny zamocowana jest w~miejscu 3. Zaletą takiego podejścia jest dokładność. Wadą jest fakt, że wskazanie zależy od
            gęstości cieczy \cite{Piotrowski2017}. 
            \begin{figure}[!htbp]
                \centering \includegraphics[width=0.6\textwidth]{poz_wypor.jpg}
                \caption{Schemat poziomierza wyporowego \cite[242]{Piotrowski2017}}
                \label{fig:poz_wypor}
            \end{figure}

            \textbf{Poziomierz wagowy} jest koncepcyjnie naprostszym rozwiązaniem, ponieważ opiera się na pomiarze masy. Urządzenia te bazują na tensometrach i~przekształcają przyłożoną
            siłę w~postaci ciężaru ciała w~sygnał elektryczny. Zaletami są wysoka dokładność, szeroki zakres pomiarowy, długowieczność oraz niska wrażliwość na zmiany temperatur \cite{Futek}.
            W~kontekście pomiarów zbiornika bezodpływowego zamontowanego pod ziemią istnieje jedna krytyczna wada czujników tensometrycznych -- problemy z~instalacją. Zbiornik na 
            odpady musi zostać umieszczony na platformie wagowej lub innej strukturze, która umożliwi pomiar ciężaru tylko zbiornika wraz z~zawartością. 
            
            Opisane wyżej poziomierze siłowe nie nadają się do monitorowania poziomu wypełnienia w~zbiornikach bezodpływowych. Wynika to z~faktu, że zawartość zbiornika nie jest jednorodna
            -- nie ma jednakowej gęstości, która wpływa między innymi na siłę wyporu nurnika, więc wskazania mogą być zniekształcone.
            Dodatkowo, na dnie szamba z~czasem gromadzą się ciężkie zanieczyszczenia -- kamień moczowy oraz osad denny, które są trudne do wypompowania \cite{eko-natural}. Ich obecność
            może negatywnie wpłynąć na dokładność wskazania poziomierza wagowego. Ponadto, zastosowanie pływaka wymaga stworzenia dodatkowej infrastruktury -- mechanizmu z~liną oraz 
            systemu przekształcającego przemieszczenie ciężarka połączonego z~pływakiem na sygnał elektryczny.

        % wywaliłem większą część z tej części nie licząc poziomierzy siłowych i wstępu
            
        \clearpage %Brzydko się łamało 
        \section{Mierniki stosowane w~zbiornikach asenizacyjnych}
            \subsection{Rejestrator poziomu Wi-Fi RU Solid IP65}
                \noindent Czujnik firmy Termipol (rys. \ref{fig:termipol_dev}) z~siedzibą w~Lublińcu. Koszt około \textbf{800 PLN}. Specyfikacja techniczna:
                \begin{enumerate}
                    \item czujnik ultradźwiękowy z~gumowym przewodem o~długości 5 m;
                    \item zakres pomiarowy: 28 –- 750 cm;
                    \item zasilanie: akumulator PCM oraz zasilacz sieciowy 5V (w zestawie);
                    \item łączność: Wi-Fi 2,4 GHz;
                    \item zasięg komunikacji: do 150 m;
                    \item czas pracy na baterii: do 365 dni;
                    \item stopień ochrony: IP65;
                    \item warunki pracy: temperatura od -25°C do 60°C, wilgotność poniżej 80\% RH;
                    \item typ pomiaru: ciągły.
                \end{enumerate}
                \begin{figure}[!htbp] 
                    \centering
                    \begin{subfigure}[b]{0.45\textwidth}
                        \centering
                        \includegraphics[width=\textwidth]{termipol_device.jpg}
                    \end{subfigure}
                    \hspace{0.05\textwidth} % odstęp między podrysunkami
                    \begin{subfigure}[b]{0.45\textwidth}
                        \centering
                        \includegraphics[width=\textwidth]{termipol_device_2.jpg}
                    \end{subfigure}
                    \caption{Czujnik poziomu Wi-Fi RU Solid IP65}
                    \label{fig:termipol_dev}
                \end{figure}

        \section{Porównanie gotowych rozwiązań} 
            \noindent Analiza dostępnych na rynku rozwiązań wykazała brak urządzeń, które spełniałyby wszystkie przyjęte na wstępie założenia. Najbardziej zaawansowany i jednocześnie oferujący bezkontaktowy pomiar
            czujnik --~RU Solid IP65 oferuje szeroki zakres pomiarowy i~zasięg rzędu 150~m, lecz czas pracy przy zasilaniu z baterii jest ograniczony, a~cena najwyższa ze wszystkich. Rozwiązania pływakowe, konduktancyjne
            i~hydrostatyczne (np. Novabo, Signax, ALERT GM-S II, SZ-02) są koncepcyjnie prostsze, jednak wymagają kontaktu z~medium, co stoi w~sprzeczności z~założeniami. W~związku z~powyższym,
            projektowany czujnik ma za zadanie wypełnić lukę rynkową: zapewnić pomiar bezkontaktowy, komunikację radiową o większym zasięgu niż Wi-Fi, niskie zużycie energii umożliwiające wieloletnią pracę na baterii
            oraz cenę konkurencyjną lub niższą niż jakiegolokwiek czujnika. Zbiorcze porównanie czujników znajduje się w \ref{tab:comparision}.
            
            \begin{table}[!htbp] \centering
                \caption{Porównanie gotowych rozwiązań}
                \label{tab:comparision}
                \small
                \begin{tabular}{l l l l l l}
                    \hline\hline
                    \textbf{\uline{Urządzenie}} & \textbf{\uline{Czujnik}} & \textbf{\uline{Interfejs}} & \textbf{\uline{Zasięg}} & \textbf{\uline{Żywotność}} & \textbf{\uline{Cena [zł]}}      \\ \hline
                    Wi-Fi RU Solid IP65         & ultradźwiękowy                & Wi-Fi 2,4 GHz              & 150 m~                  & 1 rok                       & 800                       \\  
                    SZ--02                      & hydrostatyczny                & Wi-Fi 2,4 GHz              & 150 m~                  & 4 lata                      & 599                       \\  
                    Signax                      & konduktancyjny                & radiowy 868 MHz            & 250 m~                  & $\infty$                    & 335                       \\  
                    ALERT GM-S II               & konduktancyjny                & przewodowy                 & 100 m~                  & $\infty$                    & 329                       \\  
                    Novabo\textsuperscript{a}   & \makecell[l]{pływakowy/\\konduktancyjny}   & przewodowy    & 5 m~                    & $\infty$                    & 2339                      \\  
                    \hline\hline
                \end{tabular}
                \begin{minipage}{\textwidth}
                    \footnotesize
                    \raggedright
                    \vspace{0.2cm}
                    \textsuperscript{a} Ta informacja wymaga wytłumaczenia.
                \end{minipage}
            \end{table}        

    \clearpage\chapter{Założenia projektowe}
    \noindent Na podstawie analizy istniejących rozwiązań oraz przeglądu literatury określono szereg wymagań technicznych, jakie powinno spełniać urządzenie. Wiadomo, że będzie pracowało w~środowisku zewnętrznym,
    bez dostępu do sieci energetycznej lub innych źródeł energii. Dokładność odczytów poziomu szamba nie jest kluczowa i~odchylenia rzędu centymetrów nie wpłyną na użyteczność systemu. Istotna z~punktu widzenia 
    potencjalnego nabywcy jest wygoda użytkowania, czyli urządzenie powinno być proste w~instalacji oraz niewymagające serwisowania. Na tej podstawie sformułowano następujące główne założenia funkcjonalne i~projektowe
    urządzenia:

    \begin{itemize}
        \item zasilanie bateryjne,
        \item praca na baterii: minimum 5 lat,
        \item komunikacja bezprzewodowa,
        \item zasięg komunikacji: minimum 50 metrów,
        \item dokładność pomiaru: do 10 centymetrów,
        \item koszt: do 300 złotych.
    \end{itemize}

% \section{In progess}
% clearpage \appendix{Zestawienie elementów -- BOM}
% \label{app:zalB}
% \vspace{5.5cm}
% \begin{table}[H]
%   \centering
%   \begin{adjustbox}{angle=90, max totalsize={\textheight}{\textwidth}}
%     \begin{minipage}{\textwidth} % szerokość po obrocie
%       \caption{\textcolor{red}{Przykładowe}  zestawienie elementów}
%       \label{tab:bom_main}
%       \begin{tabular}{c l~l l~l l~l}
%         \hline\hline
%         \textbf{Ilość} & \textbf{Oznaczenie} & \textbf{Part Number} &
%         \textbf{Wartość} & \textbf{Cena [PLN]} & \textbf{Producent} & \textbf{Uwagi} \\ \hline
%                 1  & Antena   & --- & ---  & 7,66  & ---         & Antena 868 MHz SMA \\
%                 1             & B1                 & CMT-8503-85B-SMT-TR & ---                    & 3,72                & Same Sky                      & Brzęczyk \\
%                 2             & C1, C2             & TMK316BJ106MLHT     & 10 $\mu$F              & 2,08                & TAIYO YUDEN                   & Kondensator MLCC \\
%                 1             & C11                & GRT188R61E475KE13D  & 4,7 $\mu$F             & 0,68                & Murata Electronics            & Kondensator MLCC \\
%                 1             & C13                & KGM15ACG1H100FT     & 10 pF                  & 0,68                & KYOCERA AVX                   & Kondensator MLCC \\
%                 1             & C20                & GJM1555C1H2R8WB01D  & 2,8 pF                 & 0,64                & Murata Electronics            & Kondensator MLCC \\
%                 2             & C21, C24           & GJM1555C1H4R5WB01D  & 3,8 pF                 & 1,28                & Murata Electronics            & Kondensator MLCC \\
%                 1             & C22                & GJM1555C1H1R3WB01D  & 1,3 pF                 & 0,60                & Murata Electronics            & Kondensator MLCC \\
%                 3             & C4, C5, C10        & GMK107B7104KAHT     & 100 nF                 & 1,20                & TAIYO YUDEN                   & Kondensator MLCC \\
%                 2             & C6, C7             & C1608X5R1E475K080AC & 4,7 $\mu$F             & 1,52                & TDK                           & Kondensator MLCC \\
%                 1             & C8                 & 885012106030        & 1 $\mu$F               & 0,40                & Wurth Elektronik              & Kondensator MLCC \\
%                 2             & C9, C12            & C1608X5R1H103K080AA & 10 nF                  & 0,80                & TDK                           & Kondensator MLCC \\
%                 1             & D1                 & LTST-C197KGKSKTCS2  & ---                    & 0,56                & Lite-On                       & Dioda LED \\
%                 1             & FB1                & BLM18KN101EH1D      & 100 $\Omega$ @ 100 MHz & 0,92                & Murata Electronics            & Koralik ferrytowy \\
%                 1             & J1, J2, J3         & G800W268018EU       & 40 pin                 & 3,11                & TE Connectivity               & Złącze szpilkowe \\
%                 1             & J4                 & SMACONNECTOR        & ---                    & 8,73                & \makecell[l]{Amphenol\\Commercial Products}  & Złącze SMA \\
%                 1             & L1                 & CDRH4D28NP-100NC    & 10 $\mu$H              & 3,12                & Sumida                        & Dławik mocy \\
%         \multicolumn{4}{r}{\textbf{Suma}} & 111,68 & \multicolumn{2}{l}{} \\
%         \hline\hline
%       \end{tabular}
%     \end{minipage}
%   \end{adjustbox}
% \end{table}
% \begin{table}[h]
%     \centering
%     \begin{adjustbox}{angle=90}
%         \begin{minipage}{\textheight} % zmieniamy na wysokość strony po obrocie
%             \caption{Zestawienie elementów -- stacja bazowa}
%             \label{tab:bom_main}
%             \begin{tabular}{c l~l l~l l~l}
%                 \hline\hline
%                 \textbf{Ilość} & \textbf{Oznaczenie} & \textbf{Part Number} & \textbf{Wartość} & \textbf{Cena [PLN]} & \textbf{Producent} & \textbf{Uwagi} \\ \hline
%                 1  & Antena   & ---                 & ---  & 7,66  & ---         & Antena 868 MHz SMA \\
%                 1             & Antena             & ---                 & ---                    & 7,66                & ---                           & Antena 868 MHz SMA \\
%                 1             & B1                 & CMT-8503-85B-SMT-TR & ---                    & 3,72                & Same Sky                      & Brzęczyk \\
%                 2             & C1, C2             & TMK316BJ106MLHT     & 10 $\mu$F              & 2,08                & TAIYO YUDEN                   & Kondensator MLCC \\
%                 1             & C11                & GRT188R61E475KE13D  & 4,7 $\mu$F             & 0,68                & Murata Electronics            & Kondensator MLCC \\
%                 1             & C13                & KGM15ACG1H100FT     & 10 pF                  & 0,68                & KYOCERA AVX                   & Kondensator MLCC \\
%                 1             & C20                & GJM1555C1H2R8WB01D  & 2,8 pF                 & 0,64                & Murata Electronics            & Kondensator MLCC \\
%                 2             & C21, C24           & GJM1555C1H4R5WB01D  & 3,8 pF                 & 1,28                & Murata Electronics            & Kondensator MLCC \\
%                 1             & C22                & GJM1555C1H1R3WB01D  & 1,3 pF                 & 0,60                & Murata Electronics            & Kondensator MLCC \\
%                 3             & C4, C5, C10        & GMK107B7104KAHT     & 100 nF                 & 1,20                & TAIYO YUDEN                   & Kondensator MLCC \\
%                 2             & C6, C7             & C1608X5R1E475K080AC & 4,7 $\mu$F             & 1,52                & TDK                           & Kondensator MLCC \\
%                 1             & C8                 & 885012106030        & 1 $\mu$F               & 0,40                & Wurth Elektronik              & Kondensator MLCC \\
%                 2             & C9, C12            & C1608X5R1H103K080AA & 10 nF                  & 0,80                & TDK                           & Kondensator MLCC \\
%                 1             & D1                 & LTST-C197KGKSKTCS2  & ---                    & 0,56                & Lite-On                       & Dioda LED \\
%                 1             & FB1                & BLM18KN101EH1D      & 100 $\Omega$ @ 100 MHz & 0,92                & Murata Electronics            & Koralik ferrytowy \\
%                 1             & J1, J2, J3         & G800W268018EU       & 40 pin                 & 3,11                & TE Connectivity               & Złącze szpilkowe \\
%                 1             & J4                 & SMACONNECTOR        & ---                    & 8,73                & \makecell[l]{Amphenol\\Commercial Products}  & Złącze SMA \\
%                 1             & L1                 & CDRH4D28NP-100NC    & 10 $\mu$H              & 3,12                & Sumida                        & Dławik mocy \\
%                 1             & L2                 & ASMPH-0603-2R2M-T   & 2,2 $\mu$H             & 1,12                & Abracon                       & Dławik mocy \\
%                 \multicolumn{4}{r}{\textbf{Suma}} & 111,68 & \multicolumn{2}{l}{} \\
%                 \hline\hline
%             \end{tabular}
%         \end{minipage}
%     \end{adjustbox}
% \end{table}


%     \begin{sidewaystable}[H] % Standardowa tabela
%         \centering
%         \footnotesize
%         \caption{Zestawienie elementów -- stacja bazowa}
%         \label{tab:bom_main}
%         \begin{tabular}{c l~l l~l l~l}
%             \hline\hline
%             \textbf{Ilość} & \textbf{Oznaczenie} & \textbf{Part Number} & \textbf{Wartość} & \textbf{Cena [PLN]} & \textbf{Producent} & \textbf{Uwagi} \\ \hline
%             1             & Antena             & ---                 & ---                    & 7,66                & ---                           & Antena 868 MHz SMA \\
%             1             & B1                 & CMT-8503-85B-SMT-TR & ---                    & 3,72                & Same Sky                      & Brzęczyk \\
%             2             & C1, C2             & TMK316BJ106MLHT     & 10 $\mu$F              & 2,08                & TAIYO YUDEN                   & Kondensator MLCC \\
%             1             & C11                & GRT188R61E475KE13D  & 4,7 $\mu$F             & 0,68                & Murata Electronics            & Kondensator MLCC \\
%             1             & C13                & KGM15ACG1H100FT     & 10 pF                  & 0,68                & KYOCERA AVX                   & Kondensator MLCC \\
%             1             & C20                & GJM1555C1H2R8WB01D  & 2,8 pF                 & 0,64                & Murata Electronics            & Kondensator MLCC \\
%             2             & C21, C24           & GJM1555C1H4R5WB01D  & 3,8 pF                 & 1,28                & Murata Electronics            & Kondensator MLCC \\
%             1             & C22                & GJM1555C1H1R3WB01D  & 1,3 pF                 & 0,60                & Murata Electronics            & Kondensator MLCC \\
%             3             & C4, C5, C10        & GMK107B7104KAHT     & 100 nF                 & 1,20                & TAIYO YUDEN                   & Kondensator MLCC \\
%             2             & C6, C7             & C1608X5R1E475K080AC & 4,7 $\mu$F             & 1,52                & TDK                           & Kondensator MLCC \\
%             1             & C8                 & 885012106030        & 1 $\mu$F               & 0,40                & Wurth Elektronik              & Kondensator MLCC \\
%             2             & C9, C12            & C1608X5R1H103K080AA & 10 nF                  & 0,80                & TDK                           & Kondensator MLCC \\
%             1             & D1                 & LTST-C197KGKSKTCS2  & ---                    & 0,56                & Lite-On                       & Dioda LED \\
%             1             & FB1                & BLM18KN101EH1D      & 100 $\Omega$ @ 100 MHz & 0,92                & Murata Electronics            & Koralik ferrytowy \\
%             1             & J1, J2, J3         & G800W268018EU       & 40 pin                 & 3,11                & TE Connectivity               & Złącze szpilkowe \\
%             1             & J4                 & SMACONNECTOR        & ---                    & 8,73                & \makecell[l]{Amphenol\\Commercial Products}  & Złącze SMA \\
%             1             & L1                 & CDRH4D28NP-100NC    & 10 $\mu$H              & 3,12                & Sumida                        & Dławik mocy \\
%             1             & L2                 & ASMPH-0603-2R2M-T   & 2,2 $\mu$H             & 1,12                & Abracon                       & Dławik mocy \\
%             1             & L3                 & LQW15AN11NJ8ZD      & 11 nH                  & 0,76                & Murata Electronics            & Cewka indukcyjna \\
%             1             & L4                 & LQW15AN16NG8ZD      & 16 nH                  & 0,84                & Murata Electronics            & Cewka indukcyjna \\
%             1             & LCD                & ---                 & ---                    & 31,97               & ---                           & Wyświetlacz LCD \\ 
%             1             & Q1                 & BSS806NEH6327XTSA1  & ---                    & 1,44                & Infineon Technologies         & Tranzystor MOSFET \\
%             1             & R10                & CRCW0603220RFKEC    & 220 $\Omega$           & 0,40                & Vishay                        & Rezystor \\
%             1             & R12                & ERJ-2RKF1000X       & 100 $\Omega$           & 0,40                & Panasonic                     & Rezystor \\
%             8             &\makecell[l]{R3, R4, R14, R15,\\R16, R17, R18, R19}&CRCW040210R0FKEDHP& 10 $\Omega$      & 3,62                          & Vishay    & Rezystor \\
%             2             & R5, R11            & RMC10-472JTH        & 4,7 k$\Omega$          & 0,80                & Kayama                        & Rezystor \\
%             7             &\makecell[l]{R6, R13,R7, R22,\\R23, R24, R25}& ERJ-2GE0R00X        & 0 $\Omega$          & 2,80                          & Panasonic& Rezystor \\
%             2             & R8, R9             & ERJ-2RKF2740X       & 274 $\Omega$           & 0,80                & Panasonic                     & Rezystor \\
%             1             & TCXO               & ECS-TXO-20CSMV4-320-AY-TR & 32 MHz           & 6,58                & ECS                           & TCXO \\
%             1             & U1                 & TPS62026DGQR        & ---                    & 3,27                & Texas Instruments             & Przetwornica Step-Down \\
%             1             & U2                 & STM32WLE5C8U6       & ---                    & 19,18               & STMicroelectronis             & Mikrokontroler \\ \hline
%             \multicolumn{4}{r}{\textbf{Suma}}                                                 & 111,68              & \multicolumn{2}{l}{} \\
%             \hline\hline
%         \end{tabular}
%     \end{sidewaystable}
%   !(na koniec należy sprawdzić BOM i~zmienić w~nim np. kondensatory -- po trymowaniu).

%     \begin{sidewaystable}[t] % Standardowa tabela
%         \centering
%         \caption{Zestawienie elementów -- moduł pomiarowy}
%         \label{tab:bom_sensor}
%         \begin{tabular}{c l~l l~l l~l}
%             \hline\hline
%             \textbf{Ilość} & \textbf{Oznaczenie} & \textbf{Part Number} & \textbf{Wartość} & \textbf{Cena [PLN]} & \textbf{Producent} & \textbf{Uwagi} \\ \hline
%             1              & Antena              & ---                  & ---              & 7,66                & ---                & Antena 868 MHz SMA \\
%             3              & C1, C2, C7          & 885012106030         & 1 $\mu$F         & 1,2                 & Wurth Elektronik   & Kondensatory MLCC \\
%             1              & C10                 & C1608X5R1H103K080AA  & 10 nF            & 0,4                 & TDK                & Kondensator MLCC \\
%             1              & C11                 & GRT188R61E475KE13D   & 4,7 $\mu$F       & 0,679               & Murata Electronics & Kondensator MLCC \\
%             1              & C13                 & KGM15ACG1H100FT      & 10 pF            & 0,679               & KYOCERA AVX        & Kondensator MLCC \\
%             1              & C20                 & GRM155R71E473JA88D   & 47 nF            & 0,4                 & Murata Electronics & Kondensator MLCC \\
%             1              & C21                 & GRM0225C1H680GA02L   & 68 pF            & 0,4                 & Murata Electronics & Kondensator MLCC \\
%             1              & C22                 & GJM1555C1H5R5CB01D   & 5,5 pF           & 0,4                 & Murata Electronics & Kondensator MLCC \\
%             1              & C23                 & GJM1555C1H2R3WB01D   & 2,3 pF           & 0,60                & Murata Electronics & Kondensator MLCC \\
%             2              & C2, C25             & GJM1555C1H3R5BB01D   & 3,5 pF           & 0,8                 & Murata Electronics & Kondensator MLCC \\
%             3              & C3, C4, C9          & GMK107B7104KAHT      & 100 nF           & 1,2                 & TAIYO YUDEN        & Kondensatory MLCC \\
%             2              & C5, C6              & C1608X5R1E475K080AC  & 4,7 $\mu$F       & 1,52                & TDK                & Kondensatory MLCC \\
%             1              & C8                  & C1608X5R1H103K080AA  & 10 nF            & 0,4                 & TDK                & Kondensator MLCC \\
%             1              & D1                  & LTST-C197KGKSKTCS2   & ---              & 0,56                & Lite-On            & Dioda LED \\
%             1              & FB1                 & BLM18KN101EH1D       & ---              & 0,92                & Murata Electronics & Koralik ferrytowy \\
%             1              & J1, J2, J3\textsuperscript{a}& G800W268018EU & ---            & 0                   & ---                & Złącze szpilkowe \\
%             1              & J6                  & SMACONNECTOR         & ---              & 8,73                & LPRS               & Złącze SMA \\
%             1              & L1                  & ASMPH-0603-2R2M-T    & 2,2 $\mu$H       & 1,12                & Abracon            & Dławik mocy \\
%             1              & L2                  & LQW15AN4N7B8ZD       & 4,7 nH           & 0,84                & Murata Electronics & Kondensator MLCC \\
%             1              & L3                  & LQW15AN4N5B8ZD       & 4,5 nH           & 0,52                & Murata Electronics & Cewka indukcyjna \\
%             1              & L4                  & LQW15AN3N6B80D       & 3,6 nH           & 0,76                & Murata Electronics & Cewka indukcyjna \\
%             1              & L5                  & LQW15AN9N2G00D       & 9,2 nH           & 0,60                & Murata Electronics & Cewka indukcyjna \\
%             1              & L7                  & LQW15AN47NG80D       & 47 nH            & 0,80                & Murata Electronics & Cewka indukcyjna \\
%             1              & R1                  & CR0603-FX-1653ELF    & 165 k$\Omega$    & 0,40                & Bourns             & Rezystor \\
%             2              & R15, R16            & ERJ-2RKF2740X        & 274 $\Omega$     & 0,80                & Panasonic          & Rezystor \\
%             1              & R2                  & CR0603-FX-2942ELF    & 29,4 k$\Omega$   & 0,40                & Bourns             & Rezystor \\
%             6              & \makecell[l]{R3, R4, R13, R14,\\ R22, R23} & ERJ-2GE0R00X     & 0 $\Omega$& 2,00    & Panasonic          & Rezystor \\
%             3              & R5, R6, R12         & RMC10-472JTH         & 4,7 k$\Omega$    & 1,20                & Kayama             & Rezystor \\
%             4              & R7, R8, R10, R11    & RMC1/16SK10R0DTH     & 10 $\Omega$      & 1,60                & Vishay             & Rezystor \\ 
%             1              & R9                  & CRCW0603220RFKEC     & 220 $\Omega$     & 0,40                & Vishay             & Rezystor \\
%             1              & TCXO                & ECS-TXO-20CSMV4-320-AY-TR & 32 MHz      & 6,58                & ECS                & TCXO \\
%             1              & U1                  & STM32WLE5C8U6        & ---              & 19,18               & STMicroelectronix  & Mikrokontroler \\
%             1              & U2                  & ADP123AUJZ-R7        & ---              & 8,30                & Analog Devices     & Regulator napięcia \\ 72,05+
%             1              & Sensor              & A02YYUW              & ---              & 109,00              & DF Robot           & Sensor odległości \\ \hline
%             \multicolumn{4}{r}{\textbf{Suma}}                                              & 181,05              & \multicolumn{2}{l}{} \\
%             \hline\hline
%         \end{tabular}
%         \begin{minipage}{\textwidth}
%             \footnotesize
%             \raggedright
%             \vspace{0.2cm}
%             \hspace{3cm}\textsuperscript{a} Złącze szpilkowe znajduje się w~BOM stacji bazowej w~tabeli \ref{tab:bom_main}.
%         \end{minipage}
%     \end{sidewaystable}

