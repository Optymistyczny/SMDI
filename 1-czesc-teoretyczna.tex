\clearpage\chapter*{Wprowadzenie}
\phantomsection\addcontentsline{toc}{chapter}{Wprowadzenie}
    \noindent Rozwój Internetu Rzeczy (IoT) jest jednym z~głównych filarów Czwartej Rewolucji Przemysłowej. Z~tego powodu inteligentne urządzenia elektroniczne zyskują coraz
    większą popularność, stopniowo dominując rynek konsumencki. Powstaje coraz więcej rozwiązań, których celem jest oszczędzanie czasu oraz
    ułatwianie użytkownikom codziennych czynności. Są to przykładowo obsługiwane zdalnie, z~poziomu telefonu, czujniki temperatury, inteligentne programatory świateł
    lub gniazdek, czy systemy zdalnej kontroli wizualnej. 
    Rozwiązania niektórych problemów pozostawiają wciąż szerokie możliwości poprawy. Jednym z~nich jest monitorowanie poziomu zapełnienia przydomowego zbiornika asenizacyjnego.
    Na terenach, do których nie została doprowadzona sieć kanalizacyjna, powszechne jest instalowanie podziemnych, bezodpływowych
    zbiorników na ścieki. Rozwiązanie to stawia przed użytkownikiem szereg dylematów, począwszy od dobrania odpowiedniego umiejscowienia, poprzez kontrolowanie
    szczelności pojemnika i~stopnia jego zapełnienia, aż do zaplanowania wywozu nieczystości. Następstwa związane z~użytkowaniem nieszczelnego lub przepełnionego
    szamba są wielopoziomowe: bakteriologiczne, toksykologiczne i~prawne. Konsekwencje natury prawnej wiążą się z~karami pieniężnymi, w~kwotach nawet do 500
    złotych. W~świetle obecnie rosnącego znaczenia ekologii, warto również zwrócić uwagę na problem, jakim jest przedostanie się do gruntu materiałów
    zakaźnych, takich jak wirusy, bakterie i~pasożyty oraz toksycznych związków chemicznych. Zatrucie agresywną chemią może być trwałe i~utrzymywać 
    się latami. Konsekwencje są tym groźniejsze, im bliżej uszkodzonego zbiornika znajdują się cieki wodne czy studnie.

    Analiza rynku konsumenckiego wykazała, że dostępne są gotowe rozwiązania opisanego problemu monitorowania poziomu zapełnienia przydomowego szamba.
    Tradycyjne podejścia najczęściej opierają się na rozwiązaniach takich jak pływaki mechaniczne czy czujniki zanurzeniowe, 
    i~są obarczone szeregiem ograniczeń. Wspomniane urządzenia pomiarowe wymagają ciągłego kontaktu z~medium w~agresywnym środowisku, co prowadzi do ich stopniowego
    zużycia i~potencjalnych awarii, a~tym samym kosztownych dla użytkownika częstych przeglądów technicznych i~wymian. Problem z~tradycyjnymi metodami pojawia się
    również podczas regularnego opróżniania zbiornika, kiedy konieczne jest wcześniejsze wyjęcie czujnika lub zastosowanie specjalnych środków ostrożności w~celu
    jego ochrony.
    Wprowadzenie bardziej zaawansowanych, bezkontaktowych metod pomiaru pozwoliłoby na poprawę komfortu użytkowników oraz na uproszczenie monitorowania przydomowego
    szamba.W~wielu branżach coraz częściej stawia się na bezprzewodowe i~energooszczędne urządzenia elektroniczne. Stosowanie
    takich rozwiązań pozwala nie tylko na automatyzację procesu, ale również na integrację systemów z~innymi urządzeniami w~ramach Internetu Rzeczy (IoT -- ang.\ \textit{Internet of Things}).
    Przegląd dostępnych rozwiązań pozwolił wysnuć wniosek, że większość istniejących urządzeń do pomiaru poziomu zapełnienia zbiornika asenizacyjnego, choć skuteczna,
    nie spełnia wszystkich wymagań gospodarstw domowych. Wady, jakimi są obarczone, to najczęściej wysokie zużycie energii lub konieczność zasilania sieciowego, co
    znacząco ogranicza miejsca, gdzie możemy je zastosować.

    Celem niniejszej pracy jest opracowanie energooszczędnego, bezprzewodowego urządzenia badającego poziom zapełnienia przydomowego zbiornika asenizacyjnego.
    Proponowane rozwiązanie wykorzystuje technologię ultradźwiękową, która, jak pokazano w~następnym rozdziale pracy, gwarantuje wysoką dokładność, bez konieczności kontaktu 
    czujnika z~medium, jakim są ścieki kanalizacyjne. Wykorzystanie zasilania bateryjnego umożliwia stosowanie urządzenia w~każdym miejscu, bez potrzeby stałego 
    podłączenia do sieci elektrycznej. Ważnym aspektem projektu jest również modularność, czyli możliwość dostosowania
    urządzenia do potrzeb użytkownika poprzez zastosowanie innego rodzaju czujników poziomu. Dzięki modularności możliwa jest także łatwa wymiana poszczególnych
    elementów systemu, co pozytywnie wpływa na obniżenie kosztów serwisowania urządzenia i~wydłużenie jego żywotności. Projekt zakłada poszukiwanie równowagi między jakością a~ceną.
    Niniejsza praca ma na celu nie tylko do opracowania funkcjonalnego
    urządzenia, ale też do wykazania, że wbrew temu co sugerują dostępne na rynku urządzenia, możliwe jest połączenie takich cech jak energooszczędność,
    przystępna cena, niezawodność oraz modularność w~jednym. Tym sposobem powstanie system, który nie tylko ułatwi obsługę szamb, ale także
    wpisze się w~globalny trend automatyzacji, optymalizacji procesów monitorowania środowiska oraz rozwoju Internetu Rzeczy.

\cleardoublepage
\part{Analiza stanu wiedzy i dobór narzędzi}
\clearpage\chapter{Przegląd technik pomiarowych i istniejących rozwiązań}
    \noindent Metody pomiaru poziomu cieczy można podzielić między innymi na  bezpośrednie i~pośrednie \cite{Piotrowski2017}.
    Metody bezpośrednie wymagają obecności osoby, która może odczytać wartość pomiaru --  odczyt jest wzrokowy. Metody pośrednie wykorzystują najczęściej jakieś urządzenie
    lub wielkość fizyczną, za~pomocą których pośrednio można zmierzyć poziom cieczy. 
    
    Poziomierze do pomiaru \textbf{bezpośredniego}:
    \begin{itemize}
        \item wziernikowe,
        \item prętowe,
        \item sondujące:
        \begin{itemize}
            \item strunowe,
            \item taśmowe.
        \end{itemize}
    \end{itemize}

    Poziomierze do pomiaru \textbf{pośredniego}:
    \begin{itemize}
        \item siłowe:
            \begin{itemize}
                \item pływakowe,
                \item wyporowe,
                \item wagowe,
            \end{itemize}
        \item ciśnieniowe:
            \begin{itemize}
                \item hydrostatyczne,
                \item piezometryczne,
            \end{itemize}
        \item elektryczne:
            \begin{itemize}
                \item oporowe,
                \item pojemnościowe,
            \end{itemize}
        \item wykorzystujące fale mechaniczne:
            \begin{itemize}
                \item akustyczne,
                \item ultradźwiękowe,
            \end{itemize}
        \item wykorzystujące fale elektromagnetyczne:
            \begin{itemize}
                \item radarowe,
                \item optyczne,
                \item jądrowe.
            \end{itemize}
    \end{itemize}

    W~ramach niniejszej pracy zdecydowano się wykorzystać metody pozwalające na autonomiczne działanie urządzenia. W~związku z~tym metody bezpośrednie w~dalszej części przeglądu
    literaturowego zostaną pominięte. 
        \section{Poziomierze siłowe} 
            \noindent\textbf{Poziomierz pływakowy} opiera swoje działanie na unoszącym się na powierzchni 
            elemencie -- pływaku, którego ruch, na przykład poprzez linę, przekłada się na przemieszczenie ciężarka lub innego przedmiotu będącego poza zbiornikiem, co pozwala uzyskać informację o~szukanym
            poziomie cieczy --~zgodnie ze schematem na rys. \ref{fig:poz_pływak}. Do ich zalet należą: prosta budowa, bardzo duże zakresy pomiarowe (do 40 metrów) i~dokładność \cite{Piotrowski2017}. Do wad należy kontakt pływaka
            z~badanym medium, co może doprowadzić z~czasem do jego degradacji. 
            \begin{figure}[!htbp]
                \centering \includegraphics[width=0.35\textwidth]{poz_pływak.jpg}
                \caption{Schemat poziomierza pływakowego \cite[241]{Piotrowski2017}}
                \label{fig:poz_pływak}
            \end{figure}

            \textbf{Poziomierz wyporowy} jest odmianą opisanego wcześniej poziomierza pływakowego. W~tym wariancie na pływak 1 (rys. \ref{fig:poz_wypor}),
            działa pewna siła wyporu związana z~zanurzeniem, która jest następnie przekazywana za pomocą dźwigni 5 lub sprężyny 2~do czujnika 4 (na przykład tensometrycznego)
            i~przeliczna na szukaną wielkość -- poziom cieczy. Podstawa sprężyny zamocowana jest w~miejscu 3. Zaletą takiego podejścia jest dokładność. Wadą jest fakt, że wskazanie zależy od
            gęstości cieczy \cite{Piotrowski2017}. 
            \begin{figure}[!htbp]
                \centering \includegraphics[width=0.6\textwidth]{poz_wypor.jpg}
                \caption{Schemat poziomierza wyporowego \cite[242]{Piotrowski2017}}
                \label{fig:poz_wypor}
            \end{figure}

            \textbf{Poziomierz wagowy} jest koncepcyjnie naprostszym rozwiązaniem, ponieważ opiera się na pomiarze masy. Urządzenia te bazują na tensometrach i~przekształcają przyłożoną
            siłę w~postaci ciężaru ciała w~sygnał elektryczny. Zaletami są wysoka dokładność, szeroki zakres pomiarowy, długowieczność oraz niska wrażliwość na zmiany temperatur \cite{Futek}.
            W~kontekście pomiarów zbiornika bezodpływowego zamontowanego pod ziemią istnieje jedna krytyczna wada czujników tensometrycznych -- problemy z~instalacją. Zbiornik na 
            odpady musi zostać umieszczony na platformie wagowej lub innej strukturze, która umożliwi pomiar ciężaru tylko zbiornika wraz z~zawartością. 
            
            Opisane wyżej poziomierze siłowe nie nadają się do monitorowania poziomu wypełnienia w~zbiornikach bezodpływowych. Wynika to z~faktu, że zawartość zbiornika nie jest jednorodna
            -- nie ma jednakowej gęstości, która wpływa między innymi na siłę wyporu nurnika, więc wskazania mogą być zniekształcone.
            Dodatkowo, na dnie szamba z~czasem gromadzą się ciężkie zanieczyszczenia -- kamień moczowy oraz osad denny, które są trudne do wypompowania \cite{eko-natural}. Ich obecność
            może negatywnie wpłynąć na dokładność wskazania poziomierza wagowego. Ponadto, zastosowanie pływaka wymaga stworzenia dodatkowej infrastruktury -- mechanizmu z~liną oraz 
            systemu przekształcającego przemieszczenie ciężarka połączonego z~pływakiem na sygnał elektryczny.

        \section{Poziomierze ciśnieniowe} 
            \noindent\textbf{Poziomierz hydrostatyczny} to typ, w~którym wysokość słupa cieczy obliczana jest na podstawie ciśnienia panującego w~zbiorniku. W~najprostszym przypadku na dnie zbiornika znajduje się
            króciec pomiarowy 1 (rys.~\ref{fig:poz_ciśnieniowe}) z~czujnikiem ciśnienia albo zastosowana jest sonda, przesyłająca sygnał do układu przetwarzania poza zbiornikiem. Na podstawie jego wskazań oraz znajomości
            ciśnienia atmosferycznego i~gęstości badanej cieczy możliwe jest obliczenie wysokości zgodnie ze wzorem:
            \begin{align}
                P~& = P_{at} + \rho \cdot g~\cdot h~
            \end{align}
            gdzie:
            \begin{itemize}
                \item $P$ -- całkowite ciśnienie,
                \item $P_{at}$ -- ciśnienie zewnętrzne wywierane na ciecz (np. atmosferyczne),
                \item $\rho$ -- gęstość cieczy,
                \item $g$ -- przyspieszenie ziemskie,
                \item $h$ -- wysokość słupa cieczy.
            \end{itemize}   
            \begin{figure}[!htbp] 
                \centering \includegraphics[width=0.75\textwidth]{poz_ciśnieniowe.jpg}
                \caption{Schemat poziomierza hydrostatycznego \cite[243]{Piotrowski2017}}
                \label{fig:poz_ciśnieniowe}
            \end{figure}
            W~przypadku zbiorników zamkniętych (ciśnieniowych) konieczne jest stosowanie różnicowych czujników ciśnienia, ponieważ kluczową kwestią dla dokładności pomiaru jest znajomość
            ciśnienia gazu lub pary znajdującego się nad cieczą. W~tym wypadku króciec pomiarowy znajduje się zarówno na dnie jak i~ponad lustrem cieczy 2. Doprowadzenie ciśnienia znad cieczy do czujnika
            odbywa się za pomocą przewodu impulsowego 3. Zaletą poziomierzy hydrostatycznych
            jest ich dokładność, stabilność oraz odporność na zanieczyszcznia \cite{Termipol-metody}. Do wad należy zależność wskazania od gęstości cieczy oraz kontakt czujnika z~medium. Należy również
            mieć na uwadze, że w~przypadku pomiaru różnicowego wewnątrz króćca górnego będzie się gromadził kondensat, który powinien zbierać się w~dodatkowym zbiorniku 4. 
        
            \textbf{Poziomierz piezometryczny} zwany jest również pneumatycznym, a~jego zasada działania polega na pomiarze ciśnienia powietrza w~rurce piezometrycznej 3 (rys. \ref{fig:piezo}), które rośnie lub maleje w~zależności
            od ciśnienia hydrostatycznego panującego na wysokości umieszczonej w~cieczy rurki pomiarowej 1. Ciśnienie hydrostatyczne jest kompensowane przez ciśnienie powietrza wtłaczanego do rurki przez kompresor albo reduktor --
            w~efekcie mierzona ciecz nie dostaje się do rurki piezometrycznej.
            Istnieje również wariant dla zbiorników zamkniętych, jednak nie jest on zalecany. W~takim wypadku wykorzystywany jest różnicowy czujnik ciśnienia 4.
            \begin{figure}[!htbp]
                \centering \includegraphics[width=0.75\textwidth]{piezo.jpg}
                \caption{Schemat poziomierza piezometrycznego \cite[245]{Piotrowski2017}}
                \label{fig:piezo}
            \end{figure}
            Zaletą tej metody jest możliwość stosowania do cieczy agresywnych. Wady to: potrzeba ciągłego dostarczania powietrza, niebezpieczeństwo zanieczyszczenia rurki piezometrycznej,
            występujący błąd systematyczny związany ze spadkiem ciśnienia w~rurce wskutek dużej ilości przepłwającego gazu oraz wpływ gęstości na wynik pomiaru \cite{Piotrowski2017}.

    \section{Poziomierze elektryczne} 
        \noindent\textbf{Poziomierze oporowe} występują w~różnych wariantach (rys. \ref{fig:poz_oporowe}). Pomiar polega na badaniu zmian konduktancji układu sonda 2 -- ścianka zbiornika 1 lub sonda 4 -- sonda 4.
        W~przypadku pierwszym konduktancja ulega zmianie, ponieważ wraz ze wzrostem poziomu cieczy skraca się długość ,,rezystora'', czyli sondy oraz ścianki zbiornika. W~drugim przypadku dwie sondy (taśmy oporowe)
        są zamknięte wewnątrz koszulki izolacyjnej 4, a~opływająca je ciecz powoduje ich zbliżenie oraz zwarcie, co przekłada się na zmianę przewodności. 
        \begin{figure}[!htbp]
            \centering \includegraphics[width=0.35\textwidth]{poz_oporowe.jpg}
            \caption{Schemat poziomierza oporowego (konduktancyjnego). Od lewej: układ z~jedną sondą, układ z~dwiema sondami \cite[246]{Piotrowski2017}}
            \label{fig:poz_oporowe}
        \end{figure}
        Zaletami są bardzo prosta budowa oraz niski koszt. Niestety rozwiązanie to nie sprawdza się w~przypadku cieczy o~dużej zmienności przewodności. Dodatkowo sondy mają kontakt z~medium.
        
        \noindent\textbf{Poziomierze pojemnościowe} działają na zasadzie zmiany pojemności układu (rys.~\ref{fig:poz_pojemnosciowe}). Wewnątrz zbiornika z~badaną cieczą znajdują się dwa elementy imitujące okładki
        kondensatora, mogą być to kombinacje: sonda -- rura, sonda -- zbiornik, sonda -- sonda. Jedna z~,,okładek'' jest zawsze pokryta izolacją elektryczną. Zmiana pojemności układu występuje wskutek zmian 
        przenikalności elektrycznej ośrodka pomiędzy okładkami kondensatora -- jeżeli poziom cieczy podnosi się, to pojemność kondensatora ulega zwiększeniu, ponieważ jest ona proporcjonalna do współczynnika
        przenikalności względnej ośrodka, zgodnie ze wzorem (\ref{kondzior}) dla idealnego kondensatora płaskiego wypełnionego dielektrykiem:
        

        \begin{align}
            \label{kondzior}
            C~& = \frac{\varepsilon \cdot S}{d}
        \end{align}
        gdzie:
        \begin{itemize}
            \item $S$ -- powierzchnia okładek kondensatora,
            \item $C$ -- pojemność kondensatora,
            \item $\varepsilon$ -- względny współczynnik przenikalności elektrycznej ,
            \item $\rho$ -- odległość okładek kondensatora.
        \end{itemize}   
        
        Zaletami poziomierzy pojemnościowych są: możliwość stosowania dla cieczy (kwasy, zasady, oleje, ciecze lepkie) i~ciał stałych oraz możliwość stosowania w~agresywnych środowiskach (wysokie ciśnienia i~temperatury). 
        Wadą są niepewność pomiaru sięgająca od 2 do 3 \% oraz dodatkowe błędy wynikające ze zmiany przenikalności dielektrycznej materiału na skutek m.in. wahań wilgotności.

        \begin{figure}[!htbp] 
            \centering \includegraphics[width=0.5\textwidth]{poz_pojemnosciowe.jpg}
            \caption{Schemat poziomierza pojemnościowego. Od lewej: czujnik cylindryczny, strunowy \cite[248]{Piotrowski2017}}
            \label{fig:poz_pojemnosciowe}
        \end{figure}

    \section{Poziomierze wykorzystujące fale mechaniczne} 
        \noindent\textbf{Poziomierze akustyczne} i~\textbf{ultradźwiękowe} działają na podobnej zasadzie. Fala mechaniczna z~nadajnika wysyłana jest w~kierunku mierzonego lustra cieczy, po czym w~wyniku gwałtownej
        zmiany gęstości ośrodka, odbija się od granicy ośrodka i~powraca do odbiornika. Różnica między jednym i~drugim rozwiązaniem polega na tym, że w~przypadku fali akustycznej zmieniana jest częstotliwość sygnału tak,
        aby po odbiciu powstała fala stojąca, co poprawia jakość odebranego sygnału. W~przypadku czujników ultradźwiękowych mierzony jest czas powrotu fali. Zaletą czujników tego rodzaju jest brak kontaktu z~cieczą,
        wysoka dokładność oraz szybkość. Do wad należą: wrażliwość na warunki atmosferyczne -- zmianę temperatury i~wilgotności oraz wysoki koszt \cite{Termipol-metody}. 
        W~związku z~istnieniem systematycznego błędu związanego z~temperaturą, przewidziane zostały rozwiązania, które pozwalają go wyeliminować. Można użyć reflektora 2, który umieszczany jest w~znanej odległości od
        sensora 1, co kompensuje zmienność prędkości fali w~funkcji temperatury. Wtedy pomiar poziomu może zostać obliczony według wzoru (\ref{ultra_a}):
        
        \begin{align}
            \label{ultra_a}
            h~& = h_0 -  \frac{c \cdot t}{2}
        \end{align}
        gdzie:
        \begin{itemize}
            \item $h$ -- wysokość słupa cieczy,
            \item $h_0$ -- odległość czujnika i~dna zbiornika,
            \item $c$ -- prędkość fali w~ośrodku,
            \item $t$ -- czas od nadania do odebrania sygnału.
        \end{itemize}   
        W~przypadku, kiedy zastosowany zostanie element odbijający falę, wzór upraszcza się do postaci (\ref{ultra_b}):
        \begin{align}
            \label{ultra_b}
            h~& = h_0 -  \frac{h_1 \cdot t}{t_1}
        \end{align}
        gdzie:
        \begin{itemize}
            \item $h$ -- wysokość słupa cieczy,
            \item $h_0$ -- odległość czujnika i~dna zbiornika,
            \item $h_1$ -- odległość czujnika i~reflektora,
            \item $t$ -- czas od nadania do odebrania sygnału,
            \item $t_1$ -- czas od nadania do odebrania sygnału przy odbiciu od reflektora.
        \end{itemize}  
        
        Umieszczenie czujnika na dnie zbiornika upraszcza równanie jeszcze bardziej (\ref{ultra_c}):
        \begin{align}
            \label{ultra_c}
            h~& =\frac{c \cdot t}{2} + h_2 
        \end{align}
        gdzie:
        \begin{itemize}
            \item $h$ -- wysokość słupa cieczy,
            \item $h_2$ -- odległość czujnika od dna zbiornika,
            \item $c$ -- prędkość fali w~ośrodku,
            \item $t$ -- czas od nadania do odebrania sygnału.
        \end{itemize}  

        \begin{figure}[!htbp] 
            \centering \includegraphics[width=0.7\textwidth]{poz_ultra.jpg}
            \caption{Poziomierze ultradźwiękowe. Od lewej: czujnik nad poziomem cieczy, z~reflektorem, czujnik wewnątrz badanego medium \cite[252]{Piotrowski2017}}
            \label{fig:poz_ultra}
        \end{figure}

        
        \section{Poziomierze wykorzystujące fale elektromangetyczne} 
            \noindent\textbf{Poziomierze radarowe} wykorzystujące mikrofale, podobnie jak ultradźwiękowe, mierzą czas nadejścia odbitej fali. Istnieją rozwiązania, w~których fala porusza się w~powietrzu -- poziomierze z~
            anteną stożkową, lub takie, w~których wykorzystany jest falowód w~postaci pręta -- poziomierze z~anteną prętową \cite{Piotrowski2017}. Fala odbija się od granicy między ośrodkami różniącymi
            się względną przenikalnością elektryczną. Ten typ czujników to tak zwane czujniki radarowe impulsowe \cite{Musz2023}. Istnieją jeszcze czujniki typu FMCW (ang. \textit{Frequency-Modulated Continuous-Wave})
            z~modulacją częstotliwościową, gdzie różnica częstotliwości sygnału nadawanego i~odbieranego w~danym momencie jest zamieniana na informację o~odległości obiektu zgodnie ze wzorem (\ref{fmcw_eq}):
            \begin{align}
                \label{fmcw_eq}
                L~& = \frac{c_{0} \cdot |\Delta  t|}{2}  =  \frac{c_0 \cdot |\Delta f|}{2 \cdot \frac{df}{dt}}
            \end{align}
            gdzie:
            \begin{itemize}
                \item $L$ -- odległość,
                \item $c_{0}$ -- prędkość światła,
                \item $\delta t$ -- różnica czasu między sygnałem odebranym i~nadawanym,
                \item $f$ -- częstotliwość fali. 
            \end{itemize}  
            \begin{figure}[!htbp] 
                \centering \includegraphics[width=0.7\textwidth]{fmcw_or.png}
                \caption{Przesunięcie częstotliwości sygnału odebranego i~nadanego w~sygnale piłokształtnym \cite{Musz2023}}
                \label{fig:fmcw}
            \end{figure}
            Zaletami czujników FMCW jest niemal zerowa strefa martwa pomiaru (porównywalna z~długością użytej fali) oraz bardzo wysoka dokładność pomiarów  \cite{WolffRadar}.
            
            \textbf{Poziomierze z~tłumieniem promieniowania} nie ogarniczają się jedynie do tłumienia promieniowania izotopowego -- może to być również fala ultradźwiękowa lub elektromagnetyczna. Tłumienie promieniowania
            izotopowego jest wykorzystywane w~bardzo niesprzyjających warunkach -- wysokiej temperaturze, czy ciśnieniu, w~przypadku ciężkich i~lepkich ośrodków oraz różnych agresywnych środowiskach. Z~uwagi na właściwości
            promieniowania gamma to właśnie ono jest najczęściej wykorzystywane, ponieważ przechodzi przez ścianki zbiorników \cite{Piotrowski2017}. Każdy system pomiarowy z~tłumieniem promieniowania składa się z~nadajnika
            i~detektora.  Na ilustracji \ref{fig:gamma}. zobrazowane zostały metody ze stałym położeniem źródła promieniowania oraz w~układzie nadążnym, gdzie położenie źródła promieniowania 1 i~detektora 3 porusza się w~górę
            w~miarę zapełniania zbiornika tak, aby moc przechodzącego promieniowania była stała, co zapewnia sprzężenie zwrotne sygnału, który trafia do przetwornika wtórnego i~regulatora 9 aż do urządzenia odczytowego pomiaru 10.
            Oba te elementy umieszczone są na ułożyskowanych śrubach pociągowych 2, które napędzane są przez kątową przekładnię zębatą 7 i~8. 
            Nadajnik i~detektor umocowane są na wózkach poruszających się wzdłóż śruby pociągowej. W~wózkach znajdują się gwintowane otwory. Pozbawione możliwości obrotu przez prowadnice 5 i~6 poruszają się góra -- dół. 
            \begin{figure}[!htbp] 
                \centering
                \begin{subfigure}[b]{0.55\textwidth}
                    \centering
                    \includegraphics[width=\textwidth]{promienieowanie_1.jpg}
                \end{subfigure}
                \hspace{0.05\textwidth} % odstęp między podrysunkami
                \begin{subfigure}[b]{0.35\textwidth}
                    \centering
                    \includegraphics[width=\textwidth]{promieniowanie_2.jpg}
                \end{subfigure}
                \caption{Zasada działania czujników z~tłumieniem promieniowania. Po lewej w~układzie ze stałym położeniem nadajnika. Po prawej w~układzie nadążnym}
                \label{fig:gamma}
            \end{figure}

        \section{Inne czujniki odległości} 
            \noindent\textbf{Czujniki triangulacyjne} opisują drugą obok czujników ToF (ang. \textit{Time of Flight}) grupę czujników. Czujniki triangulacyjne wykorzystują techniki bazujące
            na zależnościach geometrycznych. Do tej grupy należą między innymi detektory na podczerwień PSD (rys. \ref{fig:triangulacja}) emitujące światło podczerwone, które po odbiciu od badanego obiektu pada na matrycę światłoczułą,
            generującą fotoprąd i~w~zależności od~natężenia prądu wypływającego z~każdego z~wyjść pozwala ustalić położenie plamki na matrycy \cite{Matusiak2012}. Wykorzystanie tych czujników byłoby możliwe do określenia
            poziomu zapełnienia zbiornika, ale wymagałoby to elementu odbijającego światło, który unosiłby się na powierzchni cieczy. W~dodatku musiałby on zawsze znajdować się w~tym samym położeniu, żeby nie stracić
            zależności geometrycznych.
            \begin{figure}[!htbp] 
                \centering
                \begin{subfigure}[b]{0.45\textwidth}
                    \centering
                    \includegraphics[width=\textwidth]{PSD.png}
                \end{subfigure}
                \hspace{0.05\textwidth} % odstęp między podrysunkami
                \begin{subfigure}[b]{0.45\textwidth}
                    \centering
                    \includegraphics[width=\textwidth]{odbiciaaa.png}
                \end{subfigure}
                \caption{Zasada działania czujników triangulacyjnych \cite{Matusiak2012}}
                \label{fig:triangulacja}
            \end{figure}
            
        \clearpage %Brzydko się łamało 
        \section{Mierniki stosowane w~zbiornikach asenizacyjnych}
            \subsection{Rejestrator poziomu Wi-Fi RU Solid IP65}
                \noindent Czujnik firmy Termipol (rys. \ref{fig:termipol_dev}) z~siedzibą w~Lublińcu. Koszt około \textbf{800 PLN}. Specyfikacja techniczna:
                \begin{itemize}
                    \item czujnik ultradźwiękowy z~gumowym przewodem o~długości 5 m;
                    \item zakres pomiarowy: 28 –- 750 cm;
                    \item zasilanie: akumulator PCM oraz zasilacz sieciowy 5V (w zestawie);
                    \item łączność: Wi-Fi 2,4 GHz;
                    \item zasięg komunikacji: do 150 m;
                    \item czas pracy na baterii: do 365 dni
                    \item stopień ochrony: IP65;
                    \item warunki pracy: temperatura od -25°C do 60°C, wilgotność poniżej 80\% RH;
                    \item typ pomiaru: ciągły.
                \end{itemize}
                \begin{figure}[!htbp] 
                    \centering
                    \begin{subfigure}[b]{0.45\textwidth}
                        \centering
                        \includegraphics[width=\textwidth]{termipol_device.jpg}
                    \end{subfigure}
                    \hspace{0.05\textwidth} % odstęp między podrysunkami
                    \begin{subfigure}[b]{0.45\textwidth}
                        \centering
                        \includegraphics[width=\textwidth]{termipol_device_2.jpg}
                    \end{subfigure}
                    \caption{Czujnik poziomu Wi-Fi RU Solid IP65}
                    \label{fig:termipol_dev}
                \end{figure}

            \clearpage
            \subsection{Novabo}
                \noindent Czujnik firmy Novabo (rys. \ref{fig:novabo}) z~siedzibą w~Gdańsku. Koszt około \textbf{2339 PLN} (polski dystrybutor \textit{Aquatechnika Water Recycling Technology}). Specyfikacja techniczna:
                \begin{itemize}
                    \item typ: pływakowy lub konduktancyjny, hermetyczny;
                    \item długość przewodu czujnika: 5 m;
                    \item zasilanie: 230 V;
                    \item typ pomiaru: sygnalizacja.
                    \item certyfikacja CE.
                \end{itemize}
                \begin{figure}[!htbp] 
                    \centering \includegraphics[width=0.35\textwidth]{novabo_1.jpg}
                    \caption{Czujnik poziomu Novabo}
                    \label{fig:novabo}
                \end{figure}

            
            \subsection{Signax}
                \noindent Urządzenie firmy Signax (rys. \ref{fig:signax}) z~siedzibą w~Opolu. Koszt około \textbf{335 PLN}. Specyfikacja techniczna:
                \begin{itemize}
                    \item czujnik zanurzeniowy konduktancyjny;
                    \item sygnalizacja osiągnięcia określonego poziomu;
                    \item zasięg komunikacji: do 250 m~w~terenie otwartym;
                    \item częstotliwość pracy: 868 MHz;
                    \item zasilanie nadajnika: 2 baterie AA,
                    \item zasilanie odbiornika: 2 baterie AA lub opcjonalnie zasilacz sieciowy;
                    \item wymiary: 25 $\times$  16 $\times$ 12 cm;
                    \item waga: 1,3 kg;
                    \item typ pomiaru: sygnalizacja.
                \end{itemize}
                \begin{figure}[!htbp] 
                    \centering
                    \begin{subfigure}[b]{0.45\textwidth}
                        \centering
                        \includegraphics[width=\textwidth]{signax_1.jpg}
                    \end{subfigure}
                    \hspace{0.05\textwidth} % odstęp między podrysunkami
                    \begin{subfigure}[b]{0.45\textwidth}
                        \centering
                        \includegraphics[width=\textwidth]{signax_2.jpg}
                    \end{subfigure}
                    \caption{Czujnik poziomu Signax}
                    \label{fig:signax}
                \end{figure}
                
                \subsection{SZ-02}
                    \noindent Czujnik firmy \href{https://www.mojdomek.eu/products.php}{mojdomek.eu} (rys. \ref{fig:mojdomek}) z~siedzibą w~Sadach. Koszt około \textbf{599 PLN}. Specyfikacja techniczna:
                    \begin{itemize}
                        \item typ czujnika: hydrostatyczny;
                        \item zasilanie: 4 baterie AA, zapewniające do 4 lat pracy;
                        \item temperatura pracy: Od -30°C do +50°C;
                        \item łączność Wi-Fi;
                        \item powiadomienia PUSH;
                        \item aplikacja mobilna i~internetowa;
                        \item możliwość połączenia z~firmą wywożącą nieczystości;
                        \item typ pomiaru: ciągły.
                    \end{itemize}
                    \begin{figure}[!htbp] 
                        \centering
                        \begin{subfigure}[b]{0.45\textwidth}
                            \centering
                            \includegraphics[width=\textwidth]{mojdomek_1.jpg}
                        \end{subfigure}
                        \hspace{0.05\textwidth} % odstęp między podrysunkami
                        \begin{subfigure}[b]{0.45\textwidth}
                            \centering
                            \includegraphics[width=\textwidth]{mojdomek_2.jpg}
                        \end{subfigure}
                        \caption{Czujnik poziomu SZ-02}
                        \label{fig:mojdomek}
                    \end{figure}
                
                \clearpage
                \subsection{ALERT GM-S II}
                    \noindent Czujnik firmy HPD (rys. \ref{fig:alert}) z~siedzibą w~Słupnie. Koszt około \textbf{329 PLN}. Specyfikacja techniczna:
                    \begin{itemize}
                        \item zasilanie: 230 V;
                        \item typ czujnika: konduktancyjny.
                        \item długość przewodu czujnika: > 100 m;
                        \item typ pomiaru: sygnalizacja.
                    \end{itemize}
                    \begin{figure}[h!] 
                        \centering
                        \begin{subfigure}[b]{0.2\textwidth}
                            \centering
                            \includegraphics[width=\textwidth]{alert_1.png}
                        \end{subfigure}
                        \hspace{0.05\textwidth} % odstęp między podrysunkami
                        \begin{subfigure}[b]{0.65\textwidth}
                            \centering
                            \includegraphics[width=\textwidth]{alert_2.png}
                        \end{subfigure}
                        \caption{Czujnik poziomu ALERT GM-S II}
                        \label{fig:alert}
                    \end{figure}

        \section{Porównanie gotowych rozwiązań} 
            \noindent Analiza dostępnych na rynku rozwiązań wykazała brak urządzeń, które spełniałyby wszystkie przyjęte na wstępie założenia. Najbardziej zaawansowany i jednocześnie oferujący bezkontaktowy pomiar
            czujnik --~RU Solid IP65 oferuje szeroki zakres pomiarowy i~zasięg rzędu 150~m, lecz czas pracy przy zasilaniu z baterii jest ograniczony, a~cena najwyższa ze wszystkich. Rozwiązania pływakowe, konduktancyjne
            i~hydrostatyczne (np. Novabo, Signax, ALERT GM-S II, SZ-02) są koncepcyjnie prostsze, jednak wymagają kontaktu z~medium, co stoi w~sprzeczności z~założeniami. W~związku z~powyższym,
            projektowany czujnik ma za zadanie wypełnić lukę rynkową: zapewnić pomiar bezkontaktowy, komunikację radiową o większym zasięgu niż Wi-Fi, niskie zużycie energii umożliwiające wieloletnią pracę na baterii
            oraz cenę konkurencyjną lub niższą niż jakiegolokwiek czujnika. Zbiorcze porównanie czujników znajduje się w \ref{tab:comparision}.
            
            \begin{table}[!htbp] \centering
                \caption{Porównanie gotowych rozwiązań}
                \label{tab:comparision}
                \small
                \begin{tabular}{l l l l l l}
                    \hline\hline
                    \textbf{\uline{Urządzenie}} & \textbf{\uline{Czujnik}} & \textbf{\uline{Interfejs}} & \textbf{\uline{Zasięg}} & \textbf{\uline{Żywotność}} & \textbf{\uline{Cena [zł]}}      \\ \hline
                    Wi-Fi RU Solid IP65         & ultradźwiękowy                & Wi-Fi 2,4 GHz              & 150 m~                  & 1 rok                       & 800                       \\  
                    SZ--02                      & hydrostatyczny                & Wi-Fi 2,4 GHz              & 150 m~                  & 4 lata                      & 599                       \\  
                    Signax                      & konduktancyjny                & radiowy 868 MHz            & 250 m~                  & $\infty$                    & 335                       \\  
                    ALERT GM-S II               & konduktancyjny                & przewodowy                 & 100 m~                  & $\infty$                    & 329                       \\  
                    Novabo                      & \makecell[l]{pływakowy/\\konduktancyjny}   & przewodowy    & 5 m~                    & $\infty$                    & 2339                      \\  
                    \hline\hline
                \end{tabular}
            \end{table}        
                
        \section{Wybór rodzaju czujnika} 
            \noindent W~przeglądzie technik pomiarowych do monitorowania poziomu cieczy przedstawiono różnorodne podejścia: od bezpośrednich, takich jak poziomierze wziernikowe czy prętowe, po metody pośrednie,
            w~tym wykorzystujące siły wyporu, ciśnienie, właściwości elektryczne (konduktancję i pojemność), fale mechaniczne i elektromagnetyczne. W~analizie uwzględniono ich zalety i~wady. Po rozważeniu różnych
            metod zdecydowano się na zastosowanie czujników ultradźwiękowych. Czujniki ultradźwiękowe zostały wybrane ze względu na swoje kluczowe zalety:
            \begin{itemize}
                \item brak kontaktu z~medium, co zapobiega problemom związanym z~degradacją i osadzaniem zanieczyszczeń;
                \item wysoka dokładność i~szybkość działania;
                \item szeroki zakres pomiarowy;
                \item brak zależności od gęstości, czy przewodności badanej cieczy.
            \end{itemize}
            Choć ultradźwięki mogą być wrażliwe na zmiany temperatury i~wilgotności, zastosowanie kompensacji błędów, takich jak wykorzystanie wzorcowego reflektora, pozwala skutecznie poprawić ich dokładność.
            Te cechy czynią je optymalnym wyborem dla monitorowania poziomu w~zbiornikach bezodpływowych.

    \clearpage\chapter{Założenia projektowe}
    \noindent Na podstawie analizy istniejących rozwiązań oraz przeglądu literatury określono szereg wymagań technicznych, jakie powinno spełniać urządzenie. Wiadomo, że będzie pracowało w~środowisku zewnętrznym,
    bez dostępu do sieci energetycznej lub innych źródeł energii. Dokładność odczytów poziomu szamba nie jest kluczowa i~odchylenia rzędu centymetrów nie wpłyną na użyteczność systemu. Istotna z~punktu widzenia 
    potencjalnego nabywcy jest wygoda użytkowania, czyli urządzenie powinno być proste w~instalacji oraz niewymagające serwisowania. Na tej podstawie sformułowano następujące główne założenia funkcjonalne i~projektowe
    urządzenia:

    \begin{itemize}
        \item zasilanie bateryjne,
        \item praca na baterii: minimum 5 lat,
        \item komunikacja bezprzewodowa,
        \item zasięg komunikacji: minimum 50 metrów,
        \item dokładność pomiaru: do 10 centymetrów,
        \item koszt: do 300 złotych.
    \end{itemize}


