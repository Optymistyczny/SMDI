\part{Projekt i~implementacja}

\clearpage\chapter{Schemat blokowy urządzenia}

\begin{figure}[!h]
    \centering \includegraphics[width=1\textwidth]{schemat_blokowy.png}
    \caption{Schemat blokowy urządzenia}
    \label{fig:schemat_blokowy}
\end{figure}
\noindent Pierwszym krokiem projektowym było stworzenie schematu blokowego urządzenia -- rys. \ref{fig:schemat_blokowy}. Całe urządzenie składa się z~dwóch modułów: pomiarowego oraz odbiorczego,
zwanego stacją bazową. Moduł pomiarowy to urządzenie monitorujące poziom zbiornika, który przesyła informacje o~stopniu zapełnienia do stacji bazowej. Jego najważniejszymi elementami są: czujnik
ultradźwiękowy oraz mikrokontroler.
Moduł ten składa się dodatkowo z:
\begin{itemize}
    \item oscylatora TCXO 32 MHz,
    \item baterii,
    \item anteny,
    \item diod świecących.
\end{itemize}
Powodem wybrania temperaturowo kompensowanego oscylatora TCXO (ang. \textit {Temperature Compensated Crystal Oscillator}) są potencjalne wahania temperatury. TCXO według noty aplikacyjnej
\cite{ST_AN5646} jest droższym, lecz lepszym rozwiązaniem w~porównaniu do standardowego oscylatora kwarcowego -- szczególnie w~kontekście dokładności oscylacji. Dzięki TCXO można osiągnąć
mniejszy dryft (ang. \textit {drift}). Zaletą tego rozwiązania jest możliwość pracy w~węższym paśmie częstotliwości, co wiąże się ze zwiększeniem zasięgu transmisji. Mniejszy dryft powoduje, że 
środek pasma transmisyjnego znajduje się na oczekiwanej częstotliwości np. 868 MHz. W~przypadku wąskiego pasma niewielki dryft spowoduje błąd i~potencjalnie doprowadzi do rozbiegnięcia się pasm, w~których
dwa urządzenia się komunikują. 
\par Stacja bazowa, podobnie jak moduł pomiarowy, również korzysta z~mikrokontrolera oraz modułu radiowego. Jej zadaniem jest odbieranie danych wysyłanych przez moduł pomiarowy w~trybie simplex oraz
powiadamianie użytkownika o~bieżącym stanie zapełnienia zbiornika. W~tym celu wykorzystany został wyświetlacz LCD, brzęczyk oraz diody świecące. Na schemacie widoczny jest również zewnętrzny zasilacz,
który nie jest częścią tej pracy. 

\clearpage\chapter{Dobór elementów}
\noindent W~tej sekcji omówiono dobór kluczowych komponentów obu urządzeń, takich jak: czujnik ultradźwiękowy, mikrokontrolery z~modułem radiowym, oscylatory oraz anteny. 
Pozostałe elementy zostały opisane przy okazji tworzenia schematów elektrycznych.

% \section{Czujnik ultradźwiękowy}
% \noindent Obecnie na rynku można wyróżnić dwie cenowe grupy sensorów ultradźwiękowych. Pierwsza z~nich to grupa przemysłowych czujników, których koszt jest często wyższy niż 1000 złotych, dlatego 
% czujniki te nie były brane pod uwagę ze względu na wymaganą niską cenę końcową urządzenia. Przykładowi producenci czujników w~klasie przemysłowej:
% \begin{itemize}
%     \item \href{https://www.pepperl-fuchs.com/poland/pl/classid_186.htm}{Pepperl+Fuchs},
%     \item \href{https://www.ifm.com/pl/pl/category/200_010_050_001#/best/1/100}{Ifm Electronic},
%     \item \href{https://www.balluff.com/en-de/products/areas/A0001/groups/G0104}{Balluff},
%     \item \href{https://www.turck.de/en/productgroup/Sensors/Ultrasonic%20Sensors}{Turck},
%     \item \href{https://czujniki.com.pl/produkty/1-czujniki-detekcja-i-pomiar/czujniki-ultradzwiekowe,2,147}{Sels}.
% \end{itemize}

% Druga grupa to sensory pochodzące od mało znanych producentów, najczęściej z~Chin. Znaleziono jednego renomowanego producenta z~USA -- Maxbotix, który posiada certyfikowane
% sensory ultradźwiękowe, dobrą dokumentację techniczną, a~także modele CAD. Sensory można zakupić za mniej niż 200 złotych.
% Niestety, tańsze modele nie są wodoodporne ani wodoszczelne, więc nie mogą zostać wykorzystane w~środowisku zewnętrznym.

% Wśród amatorów--hobbystów elektroniki i~systemów wbudowanych najbardziej znanym i~najpowszechnieszym czujnikiem jest HC-SR04. Jego cechami charakterystycznymi są:
% \begin{itemize}
%     \item napięcie zasilania: 5 V,
%     \item średni pobór prądu": 15 mA,
%     \item zakres pomiarowy: od 2 cm do 200 cm,
%     \item wymiary: 40$\times$20$\times$15 mm,
%     \item kąt pomiaru: 15$^\circ$,
%     \item wyjście: impuls czasowy.
% \end{itemize}
% Powstało wiele klonów na bazie tego czujnika. Oferują inne zakresy pomiarowe, posiadają wyjścia cyfrowe, są dostosowane do poziomów napięć 3,3 V. Koszt tych czujników mieści się w~zakresie od 5 
% do~20 złotych. Czujniki tego typu, choć tanie, również nie są wodoodporne, dlatego nie zostały wybrane. 

% Ostatecznie zdecydowano się na czujnik \textbf{A02YYUW (SEN0311)} producenta DFRobot -- chińskiego dostawcy sprzętu typu \textit{open--source}. Spełnia on wszystkie postawione
% wymagania -- jest tani i~wodoodporny, ma niewielki średni pobór prądu, więc jest energooszczędny oraz pokrywa wymagany zakres pomiarowy. Jego specyfikacja została przedstawiona poniżej: 
% \begin{itemize}
%     \item koszt: około 100 złotych;
%     \item rozdzielczość: 1 cm;
%     \item napięcie zasilania: od 3,3 V do 5 V;
%     \item średni pobór prądu: 8 mA;
%     \item zakres pomiarowy: od 3 cm do 450 cm;
%     \item kąt pomiaru: 60$^\circ$;
%     \item temperatura pracy: od -15$^\circ$C do 60$^\circ$C;
%     \item wymiary: 63$\times$84$\times$18,5 mm;
%     \item wyjście: UART,
%     \item klasa szczelności: IP67.
% \end{itemize}

\section{Mikrokontrolery}
\label{czesc_wykonawcza:wybor_mikrokontrolera}
\noindent Przy wyborze mikrokontrolera najbardziej istotna były: energooszczędność, obsługiwane interfejsy -- UART, I2C oraz wewnętrzne peryferia -- ADC, dzięki którym można zrealizować
podłączenie różnego rodzaju czujników ultradźwiękowych. Obecnie najbardziej rozpowszechnieni producenci mikrokontrolerów zostali wymienieni poniżej. Według danych na 2021 rok kontrolowali 
oni aż 82\% globalnej sprzedaży według [\href{https://evertiq.pl/news/29525}{raportu przygotowanego przez portal IC Insights}].
\begin{itemize}
    \item NXP,
    \item Microchip,
    \item Renesas,
    \item STMicroelectronics,
    \item Infineon.
\end{itemize}

Ze względu na dużą ilość dokumentacji oraz materiałów pomocniczych, a~także doświadczenie autora, zdecydowano się na układ STMicroelectronics. Producent ten ma w~swojej ofercie układy
integrujące mikrokontroler i~moduł radiowy obsługujący system LoRa (ang.\ \textit{Long Range}), który jest obok SigFox najbardziej energooszczędnym systemem komunikacji \ref{fig:zestawienie_interfejsow}.


Rozwiązanie, w~którym mikrokontroler i~moduł radiowy są dwoma rozdzielnymi układami ma wady, ponieważ cena samego modułu radiowego to koszt
w granicach od 20 do 30 złotch, gdzie koszt układów integrujących moduł radiowy jest podobny. Wiązałoby się to ze zwiększeniem skomplikowania i~gabarytów obwodu drukowanego.

STMicroelectronics posiada jedną rodzinę mikrokontrolerów wspierających komunikację bezprzewodową i~jest to \textbf{STM32WL}.
W jej ramach można wyróżnić kilka podgrup układów zgodnie z~rys. \ref{fig:zestawienie_rodziny_wl}.

\begin{figure}[H]
    \centering \includegraphics[width=1\textwidth]{zestawienie_interfejsow.png}
    \caption{\href{https://neuronicworks.com/blog/wireless-communication-protocols/}{Zestawienie interfejsów według portalu \textit{neuronicworks.com}}}
    \label{fig:zestawienie_interfejsow}
\end{figure}

\begin{figure}[H]
    \centering \includegraphics[width=1\textwidth]{zestawienie_rodziny_wl.png}
    \caption{\href{https://www.st.com/en/microcontrollers-microprocessors/stm32wl-series.html}{Rodzina STM32WL od \textit{STMicroelectronics}}}
    \label{fig:zestawienie_rodziny_wl}
\end{figure}

W projekcie założona została komunikacja punkt--punkt (ang. \textit{peer--to--peer}), jednak zapewniono możliwość wykorzystania LoRaWAN w~razie potrzeby rozbudowy i~zwiększenia zasięgu systemu.
Stąd też rozważane były mikrokontrolery, które wspierają przytoczony wcześniej intefejs, czyli posiadające oznaczenie STM32WLx5xx (rys. \ref{fig:zestawienie_rodziny_wlx5xx}).
\href{https://www.thethingsnetwork.org/docs/lorawan/what-is-lorawan/}{LoRaWAN to protokół sieciowy, który posiada własną,
odrębną infrastrukturę~-- serwery, bramki i~urządzenia końcowe. Dzięki niemu można połączyć urządzenia korzystające z~techniki modulacji jaką jest LoRa będące w~różnych miejscach na świecie.} 

\begin{figure}[!h]
    \centering
    \begin{subfigure}[b]{0.45\textwidth}
        \centering
        \includegraphics[width=\textwidth]{zestawienie_rodziny_wl55.png}
    \end{subfigure}
    \hspace{0.05\textwidth} % odstęp między podrysunkami
    \begin{subfigure}[b]{0.45\textwidth}
        \centering
        \includegraphics[width=\textwidth]{zestawienie_rodziny_wle5.png}
    \end{subfigure}
    \caption{Układy wspierającyce LoRaWAN. Po lewej \href{https://www.st.com/en/microcontrollers-microprocessors/stm32wlex.html}{WL55}.
    Po prawej \href{https://www.st.com/en/microcontrollers-microprocessors/stm32wl5x.html}{WLE5}}
    \label{fig:zestawienie_rodziny_wlx5xx}
\end{figure}

Ostatecznie wybrany został STM32WLE5C8U6 -- wróżniony na rys. \ref{fig:zestawienie_rodziny_wlx5xx}. Został użyty zarówno w~stacji bazowej, jak i~w~module pomiarowym.
Ma jeden rdzeń Cortex M4 oraz najmniejszą ilość pamięci z~dostępnych możliwych: 65 KB flash i~20 KB RAM, co
\href{https://www.mouser.pl/ProductDetail/STMicroelectronics/STM32WLE5C8U6?qs=DPoM0jnrROVd%252B5XuoleJMQ%3D%3D}{przekłada się na jego niską cenę około 18~złotych}.
Dodatkową zaletą jest to, że w~ramach serii WLE5 w~tej samej obudowie znajdują się jeszcze dwa układy, które posiadają kolejno 128 KB/48 KB i~256 KB/64 KB pamięci flash i~RAM,
co może być przydatne w~przypadku zwiększonego zapotrzebowania pamięci w~przyszłości. Seria WL55 została odrzucona, ponieważ posiada tylko jedną wersję pamięciową.
 
% \section{Oscylatory}
% \noindent Zgodnie z~notą aplikacyjną \cite{ST_AN5646} moduł radiowy wymaga do swojego działania oscylatora kwarcowego 32 MHz o~odpowiedniej dokładności (rys. \ref{fig:ppm_vs_bw}), od której 
% zależy to w~jakim paśmie BW (ang. \textit{bandwidth}) może być modulowany sygnał, co wpływa na szybkość przesyłu informacji, pobór energii oraz zasięg transmisji.
% \begin{figure}[!h]
%     \centering \includegraphics[width=1\textwidth]{ppm_vs_bw.png}
%     \caption{Minimalna dokładność oscylatora w~ppm w~stosunku do wykorzystanego pasma modulacji}
%     \label{fig:ppm_vs_bw}
% \end{figure}
% Żeby móc wykorzystać potencjał komunikacji LoRa, zdecydowano się zapewnić możliwość pracy w~bardzo wąskich pasmach modulacji w~celu wydłużenia zasięgu transmisji.
% Dlatego wybrano oscylator kompensowany temperaturowo \textit{TCXO} (ang. \textit{Temperature Compensated Cristal Oscillator}). Układ został wykorzystany w~obu 
% urządzeniach, czyli w~stacji bazowej i~module pomiarowym, ponieważ zarówno odbiornik i~nadajnik muszą spełniać wymagania co do dokładności oscylacji. Teoretycznie
% można użyć standardowych oscylatorów kwarcowych, jednak producent zaleca w~takich przypadkach większą ostrożność przy projektowaniu PCB -- użycie barier termicznych
% oraz odpowiednich pojemności szeregowych, jednak należy wtedy przeprowadzić badania dryftu częstotliwości dla skrajnych temperatur pracy układu zgodnie z\cite{ST_AN5646}. 

% \href{https://www.mouser.pl/ProductDetail/ECS/ECS-TXO-20CSMV4-320-AY-TR?qs=Jm2GQyTW%2FbhklRtAEeJs7A%3D%3D&srsltid=AfmBOooeSpJQeVHP5OJGR43soBQrlJaRoepbbev2RtHKFU5CmzG-BHa4}{ECS-TXO-20CSMV4-320-AY-TR} 
% o dokładności $\pm$ 1 ppm. Specyfikacja:
% \begin{itemize}
%     \item wyjście: \textit{clipped sine wave} od 0,8 V;
%     \item zasilanie: od 1,7 V do 3,6 V;
%     \item pobór prądu: 2 mA;
%     \item obciążenie wyjścia: 10 k$\Omega$ // 10 pF;
%     \item czas startu: 2 s.
% \end{itemize}

% W~tabeli \ref{tab:zasieg_lora}. znajduje się porównanie wykorzystanego pasma i~zasięgu. Wartości pochodzą z~internetowego kalkulatora  producenta Semtech dla układu SX1272.
% Do obliczeń zostały przyjęte następujące najważniejsze parametry komunikacji:
% \begin{itemize}
%     \item moc nadajnika 14 dBm;
%     \item SF = 8 (ang. \textit{spreading factor});
%     \item CR = 4/5 (ang. \textit{coding rate});
%     \item rozmiar danych 10 B.
% \end{itemize}

% \begin{table}[!h] \centering
%     \caption{\href{https://www.semtech.com/design-support/lora-calculator}{Zasięg komunikacji przez LoRa w~zależności od wykorzystanego pasma modulacji BW}}
%     \label{tab:zasieg_lora}
%     \begin{tabular}{c  c}
%         \hline
%         \uline{BW [kHz]} & \uline{Zasięg [km]}  \\
%         8                & 4,59                 \\ 
%         125              & 2,09                 \\ 
%         250              & 1,72                 \\ 
%         500              & 1,24                 \\ 
%         \hline
%     \end{tabular}
% \end{table}

% \section{Anteny}
% \noindent Z~uwagi na wybór interfejsu LoRa należało dobrać odpowiednią antenę. LoRa pracuje w~paśmie częstotliwości nielicencjonowanych ISM (ang. \textit{Industrial, Scientific, Medical}). 
% \href{https://pl.farnell.com/podstawy-lorawan}{W Europie jest to konkretnie częstotliwość 868 MHz}. Zdecydowano się na anteny dedykowane dla systemów LoRa z~popularnym złączem
% SMA (ang. \textit{Sub Miniature version A}), takie same dla obu urządzeń. 
% \href{https://kamami.pl/anteny-i-akcesoria/1185255-lpwa-antenna-2dbi-gain-lora-antenna-options-for-frequency-version-5906623423774.html}
% {Specyfikacja dobranej anteny od Waveshare (P/N 1185255)}:
% \begin{itemize}
%     \item częstotliwość pracy: od 833 do 903 MHz;
%     \item zysk energetyczny: 2 dBi;
%     \item złącze: wtyk męski SMA;
%     \item polaryzacja: pionowa;
%     \item typ anteny: wielokierunkowa;
%     \item impedancja: 50 $\Omega$;
%     \item SWR < 2,0;
%     \item długość 110 mm.
% \end{itemize}

% \section{Wyświetlacz}
% \noindent W~przypadku wyświetlacza istotne były: jego cena, rozmiar, wbudowany kontroler oraz dostępność zewnętrznych bibliotek. Zdecydowano się na tani wyświetlacz dotykowy TFT 
% (ang. \textit{thin-field transistor}) LCD, sprowadzany z~Chin od nieznanego producenta. Specyfikacja:
% \begin{itemize}
%     \item matryca: TFT 2,4";
%     \item rozdzielczość: 320$\cdot$240 pikseli;
%     \item głębia kolorów: 16 bitów;
%     \item kontroler: ILI9341;
%     \item interfejs: SPI;
%     \item zasilanie: 3,3 V;
%     \item slot na kartę SD;
% \end{itemize}

\clearpage\chapter{Schematy elektrycze, obwody drukowane, obliczenia i~symulacje}

\section{Wybór narzędzi} %\vspace{-1\baselineskip} % Zmniejsza odstęp
    \noindent Przed przystąpieniem do projektowania urządzenia niezbędne było dokonanie przeglądu dostępnych narzędzi projektowych. Politechnika Warszawska zapewnia
    odpowiednie oprogramowanie przystosowane do tego typu zadań. W~toku studiów wykorzystywane były m.in. takie programy jak Altium Designer, LTSpice, Qucs oraz Autodesk Inventor.
    Do zaprojektowania schematów oraz obwodów drukowanych wybrany został Altium Designer. LTSpice pozwolił na prostą weryfikację działania poszczególnych układów elektronicznych.
    Qucs umożliwił zasymulowanie obwodów wysokiej częstotliwości~-- w~szczególności możliwe było oszacowanie dopasowania toru radiowego do anteny i~na odwrót.
    Ostatnim krokiem projektowania urządzenia było stworzenie obudowy, do czego wykorzystano program Autodesk Inventor. W~tej części pracy opisano proces tworzenia
    schematów elektrycznych i~powstawania obwodów drukowanych, dobór istotnych elementów oraz niezbędne obliczenia i~symulacje.

\section{Schematy urządzeń} %\vspace{-1\baselineskip} % Zmniejsza odstęp

    \subsection{Stacja bazowa}
    \noindent W~sekcji mikrokontrolera  (rys. \ref{fig:main_mcu}) znajdują się wyprowadzenia najważniejszych sygnałów: interfejs 
    wyświetlacza LCD (\textbf{SPI\_NSS, SPI\_SCK, SPI\_MOSI, SPI\_MISO, LCD\_RES, DIM, DCx}), sygnał PWM sterujący brzęczykiem (\textbf{BUZZ}), wyjścia ,,push--pull'' sterujące diodami LED (\textbf{LED1, LED2})
    oraz sygnały radiowe  \textbf{RFI\_P} i~\textbf{RFI\_N} przesunięte względem siebie o~180$^\circ$. PH3\_BOOT0 służy do określenia trybu, w~jakim nastąpi \textit{boot} (ang. wybudzenie)
    układu \cite[18]{ST_DS13105}. Stan niski oznacza wybudzenie z~pamięci FLASH użytkownika \cite[59]{ST_RM0461}. Dokładny opis wykorzystanych wyprowadzeń znajduje się w~tabeli \ref{tab:base_pins}.
    
    \begin{figure}[H]
        \centering \includegraphics[width=1\textwidth]{main_mcu.png}
        \caption{Stacja bazowa -- sekcja mikrokontrolera}
        \label{fig:main_mcu}
    \end{figure}

    W przypadku sekcji zasilania \ref{fig:main_pwr} należy zaznaczyć, że projektowany układ może być zasilany na dwa sposoby: poprzez liniowy stabilizator napięcia (LDO) lub poprzez przetwornicę impulsową
    typu buck (SMPS). Schemat blokowy zasilania został przedstawiony na rysunku~\ref{fig:ldo_smps}. Zalety i wady zarówno LDO jak i SMPS odpowiadają tym spotykanym w standardowych układach. 
    LDO zapewnia niższy poziom szumów, jednak jego sprawność energetyczna spada przy dużej różnicy napięć między wejściem a~wyjściem. Z~kolei przetwornica SMPS pozwala na pobór większych prądów i~uzyskanie
    wyższej sprawności \cite[23]{ST_DS13105}, lecz wymaga zastosowania dodatkowych elementów pasywnych (np. cewek indukcyjnych) oraz generuje większe zakłócenia elektromagnetyczne.  Wybór trybu zasilania 
    jest konfigurowany programowo. Domyślnie, po każdym resecie układu wybierany, jest stabilizator LDO, natomiast tryb SMPS ma priorytet, jeśli zostanie odpowiednio aktywowany przez użytkownika. Wybór
    trybu zasilania, czyli zasilanie z~przetwornicy SMPS albo z~LDO jest programowy. W~przypadku trybu SMPS należy użyć odpowiedniej cewki indukcyjnej. LDO jest domyślnie wybierane każdorazowo po resecie
    zasilania. Istotne jest, aby przed uruchomieniem SMPS zapewnić detekcję zegara umożliwiającego pracę przetwornicy. W~przeciwnym razie układ może ulec uszkodzeniu \cite[26]{ST_DS13105}.  

    \begin{figure}[H]
        \centering \includegraphics[width=1\textwidth]{ldo_smps.png}
        \caption{Typy zasilania mikrokontrolera}
        \label{fig:ldo_smps}
    \end{figure}

    Podsekcja z~TCXO została zaprojektowana na podstawie komercyjnego układu (Nucleo-WL5JC) znalezionego w~dokumentacji STMicroelectronics \cite[5]{ST_MB1389}. Wartości pojemności
    oraz rezystancji w~obwodzie wyjściowym zostały zaczerpnięte z~dokumentacji technicznej przyjętego w~projekcie mikrokontrolera \cite[96]{ST_DS13105}.
    % \begin{figure}[H]
    %     \centering \includegraphics[width=0.4\textwidth]{tcxo_circuitry.png}
    %     \caption{Układ TCXO na płytce ewaluacyjnej Nucleo-WL5JC \cite{ST_MB1389}}
    %     \label{fig:tcxo_circuitry}
    % \end{figure}

    \begin{table}[H] \centering
        \caption{Opis wyprowadzeń i~sygnałów w~stacji bazowej}
        \label{tab:base_pins}
        \begin{tabular}{l l l l}
            \hline\hline
            \textbf{\uline{Terminal}}     &  \textbf{\uline{Sygnał}}&   \textbf{\uline{Typ}} &  \textbf{\uline{Przeznacznie}}           \\
            PB4                      & DCx                      & push-pull              & LCD: Wybór dane / komenda                    \\  
            PB6                      & LCD\_RES                 & push-pull              & LCD: reset                                   \\ 
            PA2                      & DIM                      & PWM                    & LCD: kontrola jasności ekranu                \\ 
            PA3                      & BUZZ                     & push-pull              & Sygnał sterujący brzęczykiem                 \\ 
            PA4                      & SPI\_NSS                 & alternatywny           & LCD: zezwolenie na komunikację               \\ 
            PA5                      & SPI\_SCK                 & alternatywny           & LCD: sygnał zegarowy                         \\ 
            PA6                      & SPI\_MISO                & alternatywny           & LCD: dane przychodządze                      \\ 
            PA7                      & SPI\_MOSI                & alternatywny           & LCD: dane wychodzące                         \\ 
            PH3-BOOT0                & ---                      & wejście                & wybór trybu wybudzenia                       \\ 
            RFI\_P                   & RFI\_P                   & alternatywny           & RF: wejście P~                               \\ 
            RFI\_N                   & RFI\_N                   & alternatywny           & RF: wejście N~                               \\ 
            PA10                     & LED2                     & push-pull              & sterowanie diodą 2                           \\ 
            PA11                     & SUBGHZ\_RES              & alternatywny           & RF: sygnał reset radia                       \\ 
            PA12                     & LED1                     & push-pull              & sterowanie diodą 1                           \\ 
            PA13                     & SWDIO                    & alternatywny           & Programator: dane                            \\ 
            PA14                     & SWCLK                    & alternatywny           & Programator: sygnał zegarowy                 \\ 
            PB0-VDD\_TCXO            & VDDTCXO                  & alternatywny           & TCXO: zasilanie                              \\ 
            OSC\_IN                  & ---                      & alternatywny           & TCXO: sygnał zegarowy z~TCXO                 \\ \hline
            VDD                      & ---                      & zasilanie              & zasilanie cz. cyfrowej układu                \\ 
            VBAT                     & ---                      & zasilanie              & zastępcze zasilanie (RTC, TAMP etc)          \\ 
            VDDSMPS                  & ---                      & zasilanie              & zasilanie  przetwornicy SMPS                 \\ 
            VDDRF                    & ---                      & zasilanie              & RF: zasilanie bloku radiowego                \\ 
            VDDA                     & ---                      & zasilanie              & zasilanie cz. analogowej układu              \\ 
            VLXSMPS                  & ---                      & zasilanie              & wyjście przetwornicy SMPS                    \\ 
            VFBSMPS                  & ---                      & zasilanie              & pętla sprzężenia zwrotnego do SMPS           \\ 
            VDDRF1V55                & ---                      & zasilanie              & RF: zasilanie bloku radiowego przez LDO/SMPS \\ 
            VDDPA                    & ---                      & zasilanie              & RF: zasilanie wzmacniacza mocy PA            \\ 
            NRST                     & nRST                     & zasilanie              & sygnał resetu (aktywny stan niski)           \\ 
            VSSSMPS                  & ---                      & zasilanie              & masa układu                                  \\ 
            VSS                      & ---                      & zasilanie              & masa układu                                  \\ 
            \hline\hline
        \end{tabular}
    \end{table}
    \newpage

    
    \begin{table}[H] \centering
        \caption{Opis wyprowadzeń i~sygnałów -- moduł pomiarowy}
        \label{tab:sensor_pins}
        \begin{tabular}{l l l l}
            \hline\hline
            \textbf{Terminal}        &  \textbf{Sygnał}         &   \textbf{Konfiguracja}        &  \textbf{Przeznacznie}                         \\ \hline
            PB3                      & LED1                     & push-pull                      & sterowanie diodą 1                             \\  
            PB4                      & LED2                     & push-pull                      & sterowanie diodą 2                             \\  
            PA4                      & SPI\_NSS                 & alternatywny                   & zezwolenie na komunikację                      \\  
            PA5                      & SPI\_SCK                 & alternatywny                   & sygnał zegarowy                                \\  
            PA6                      & SPI\_MISO                & alternatywny                   & LCD: dane przychodzące                         \\  
            PA7                      & SPI\_MOSI                & alternatywny                   & LCD: dane wychodzące                           \\  
            PA9\textsuperscript{a}   & SENSOR\_2                & alternatywny                   & I2C1\_SCL || USART1\_TX                        \\  
            PH3-BOOT0                & ---                      & wejście                        & wybór trybu wybudzenia                         \\ 
            RFO\_LP                  & RFO\_LP                  & alternatywny                   & RF: wyjście LP                                 \\  
            RFO\_HP                  & RFO\_HP                  & alternatywny                   & RF: wyjście HP                                 \\  
            VR\_PA                   & VR\_PA                   & alternatywny                   & RF: wyjście wzmacniacza mocy PA                \\  
            PA10\textsuperscript{b}  & SENSOR\_1                & alternatywny                   & I2C1\_SDA || USART1\_RX || ADC                 \\  
            PA11                     & RES\_OUT                 & alternatywny                   & RF: wyjście reset\textsuperscript{c}           \\  
            PA13                     & SWDIO                    & alternatywny                   & Programator: dane                              \\ 
            PA14                     & SWCLK                    & alternatywny                   & Programator: sygnał zegarowy                   \\ 
            PB0-VDD\_TCXO            & VDDTCXO                  & alternatywny                   & TCXO: zasilanie                                \\ 
            OSC\_IN                  & ---                      & alternatywny                   & TCXO: sygnał zegarowy z~TCXO                   \\ \hline
            VDD                      & ---                      & zasilanie                      & zasilanie cz. cyfrowej układu                  \\ 
            VBAT                     & ---                      & zasilanie                      & zastępcze zasilanie (RTC, TAMP etc)            \\ 
            VDDSMPS                  & ---                      & zasilanie                      & zasilanie  przetwornicy SMPS                   \\ 
            VDDRF                    & ---                      & zasilanie                      & RF: zasilanie bloku radiowego                  \\ 
            VDDA                     & ---                      & zasilanie                      & zasilanie cz. analogowej układu                \\ 
            VLXSMPS                  & ---                      & zasilanie                      & wyjście przetwornicy SMPS                      \\ 
            VFBSMPS                  & ---                      & zasilanie                      & pętla sprzężenia zwrotnego do SMPS             \\ 
            VDDRF1V55                & ---                      & zasilanie                      & RF: zasilanie bloku radiowego przez LDO/SMPS   \\ 
            VDDPA                    & ---                      & zasilanie                      & RF: zasilanie wzmacniacza mocy PA              \\ 
            NRST                     & nRST                     & zasilanie                      & sygnał resetu (aktywny stan niski)             \\ 
            VSSSMPS                  & ---                      & zasilanie                      & masa układu                                    \\ 
            VSS                      & ---                      & zasilanie                      & masa układu                                    \\             
            \hline\hline
        \end{tabular}
        
        \begin{minipage}{\textwidth}
            \footnotesize
            \raggedright
            \vspace{0.2cm}
            \textsuperscript{a,b} PA9 oraz PA10 służą komunikacji z~sensorem ultradźwiękowym. \\
            \textsuperscript{c} Wyjście pozwalające sprawdzić, czy moduł radiowy jest w~stanie resetu.
        \end{minipage}
    \end{table}
  
    W~przypadku stacji bazowej, która nie ma tak krytycznych wymagań dotyczących zużycia energii, zdecydowano się użyć przetwornicy typu \textit{step-down} z~5 V na 3,3 V (rys. \ref{fig:main_pwr}).
    Przyjęto 5 V jako napięcie wejściowe z~powodu większej ich popularności oraz oferowanej większej mocy, co mogłoby okazać się przydatne w~przypadku rozbudowy projektu.
    
    Przy projektowaniu układu zasilania wzięto pod uwagę wskazówki dotyczące ilości i~rozmieszczenia kondensatorów filtrujących znajdujące się w~dokumentacji mikrokontrolera zgodnie
    z~rys. \ref{fig:pwr_supply_scheme}. 


    \begin{figure}[H]
        \centering \includegraphics[width=1\textwidth]{main_pwr.png}
        \caption{Stacja bazowa -- układ zasilania}
        \label{fig:main_pwr}
    \end{figure}

    \begin{figure}[H]
        \centering \includegraphics[width=0.8\textwidth]{pwr_supply_scheme.png}
        \caption{Zalecany układ zasilania. Dokumentacja techniczna STM32WLE5C8U6 \cite[62]{ST_DS13105}}
        \label{fig:pwr_supply_scheme}
    \end{figure}

    \clearpage
    Przy złączu programatora widoczny jest rezystor podciągający podłączony do sygnału nRST, czyli mikrokontroler domyślnie jest w~trybie pracy. Rezystory w~liniach SWDIO i~SWCLK 
    mają dwojakie przeznaczenie -- umożliwiają proste sprawdzenie stanu linii lub komunikacji np. oscyloskopem oraz mogą zostać użyte do ograniczenia prądu i~kontroli czasu narastania zboczy. 
    
    \begin{figure}[H]
        \centering \includegraphics[width=0.7\textwidth]{main_debug_header.png}
        \caption{Stacja bazowa -- złącze programatora}
        \label{fig:main_debug_header}
    \end{figure}

    Ilustracja \ref{fig:main_lcd_header} ukazuje złącze wyświetlacza LCD. Rezystor \textit{pull--up} (ang. podciągający) R22 na stałe wyłącza sygnał resetu.
    Rezystor R13 zapewnia maksymalną jasność ekranu, ale w~przypadku chęci oszczędzania energii można go wymontować i~sterować linią sygnałem PWM. R22 podciąga linię reset ekranu, co zapewnia
    nieprzerwaną pracę wyświetlacza. Rezystor R23 zapewnia stały stan niski na linii SPI\_NSS, co oznacza, że komunikacja z~wyświetlaczem jest cały czas aktywna.
   
    \begin{figure}[H]
        \centering \includegraphics[width=0.7\textwidth]{main_lcd_header.png}
        \caption{Stacja bazowa -- złącze wyświetlacza LCD}
        \label{fig:main_lcd_header}
    \end{figure}

    Na rys. \ref{fig:main_bzzzzzzzzzzt} można zobaczyć brzęczyk sterowany sygnałem PWM przez klucz tranzystorowy Q1. Rezystor ściągający R11 został dodany na wypadek, gdyby wyjście mikrokontrolera 
    było w~stanie pływającym (ang. \textit{floating}), żeby bramka mogła ulec rozładowaniu, a~brzęczyk wyłączył się.

    \begin{figure}[H]
        \centering \includegraphics[width=1\textwidth]{main_bzzzzzzzzzzt.png}
        \caption{Stacja bazowa -- sekcja brzęczyka}
        \label{fig:main_bzzzzzzzzzzt}
    \end{figure}

\subsection{Moduł pomiarowy}
\noindent
    \begin{figure}[H]
        \centering \includegraphics[width=1\textwidth]{sensor_mcu.png}
        \caption{Moduł pomiarowy -- sekcja mikrokontrolera}
        \label{fig:sensor_mcu}
    \end{figure}

    Zmiana mocy transmisji odbywa się w następujący sposób: w pierwszej kolejności należy umieścić odpowiednią zworę w~torze transmisyjnym
    (rezystory R22 oraz R23, rys. \ref{fig:sensor_rf}) oraz odpowiednio zasilić terminal VDDPA \cite[20]{ST_DS13105}:
    \begin{itemize}
        \item zasilenie REG PA z~VDD (3,3~V): +22 dBm;
        \item zasilenie REG PA z~VFBSMPS (1,55~V): +15 dBm.
    \end{itemize}

    \begin{figure}[H]
        \centering \includegraphics[width=1\textwidth]{sensor_pwr.png}
        \caption{Moduł pomiarowy -- układ zasilania}
        \label{fig:sensor_pwr}
    \end{figure}
    \begin{figure}[H]
        \centering \includegraphics[width=0.7\textwidth]{sensor_debug_header.png}
        \caption{Moduł pomiarowy -- złącze programatora}
        \label{fig:sensor_debug_header}
    \end{figure}
    \begin{figure}[H]
        \centering \includegraphics[width=0.7\textwidth]{sensor_sensor_itf.png}
        \caption{Moduł pomiarowy -- złącze sensora}
        \label{fig:sensor_sensor_itf}
    \end{figure}


    \clearpage
    \vspace{-1\baselineskip}
    \subsection{Tor nadawczy i~odbiorczy}
    \vspace{-0.5\baselineskip}
    \noindent Projekt toru radiowego oparto w~całości na nocie aplikacyjnej STMicroelectronics \cite{ST_AN5457}. Każdy fragment toru radiowego posiada impedancję charakterystyczną na poziomie 50
    $\Omega$ w~celu dopasowania impedancji wyprowadzenia mikrokontrolera oraz anteny i~zminimalizowania tym samym strat w~torze sygnałowym \cite[6]{ST_AN5457}. 
    Na ilustracji \ref{fig:rf_tx_diagram}. widoczny jest schemat ideowy toru nadawczego wykorzystanego w~projekcie. Na wyjściu wzmacniacza mocy PA jest znajduje się tranzystor MOS, któru generuje
    wysokoczęstotliwościowy sygnał PWM, którego częstotliwość stanowi częstotliwość podstawową sinusoidalnego sygnału wyjściowego. LoRa jest techniką modulacji opartą o~sygnał typu \textit{chirp},
    czyli sygnał o~zmiennej w~sposób ciągły częstotliwości w~czasie. Efekt zmiany częstotliwości sygnału sinusoidalnego osiąga się, zmieniając częstotliwość sygnału PWM. PWM należy przed transmisją
    przefiltrować. Szybkie przełączanie tranzystorów wytwarza pożądany sygnał, który jest potem przekształcany przez układ filtrów i~trafia do anteny skąd jest emitowany w~przestrzeń.
   
    \begin{figure}[H]
        \centering \includegraphics[width=0.7\textwidth]{rf_tx_diagram.png}
        \caption{Schemat ideowy toru nadawczego wg. \cite[15]{ST_AN5457}}
        \label{fig:rf_tx_diagram}
    \end{figure}

    \begin{figure}[H]
        \centering \includegraphics[width=1\textwidth]{rf_tx.png}
        \caption{Typowy tor nadawczy wg. \cite[20]{ST_AN5457}}
        \label{fig:rf_tx}
    \end{figure}

    W~dokumentacji przedstawione zostały metody analityczne obliczenia wartości dla poszczególnych elementów toru. Obliczenia:
    Pojemność wyjściowa:
    \begin{align}
        C_{\text{out}} & = \frac{1}{2 \pi f~\cdot X_C} = \frac{1}{2 \pi \cdot 868 \cdot 10^6 \, \text{Hz} \cdot 1.27 \, \Omega}
    \end{align}


    Indukcyjność L1
    \begin{align}
        m~& = \frac{50}{15.27} - 1 = 1.508 \\
        L_1 & = \frac{1}{2 \pi f~\cdot 50} \cdot \frac{m}{m^2 + 1 + X_C} \\
        L_1 & = \frac{1}{2 \pi \cdot 868 \cdot 10^6} \cdot \frac{1.508}{1.508^2 + 1 + 1.27} \approx 4.45 \, \text{nH}
    \end{align}


    
    Pojemność C1
    \begin{align}
        C_1 & = \frac{1}{2 \pi f~\cdot \frac{R_{\text{out}}}{50}} \cdot \left(1 + m^2 - \frac{1 + m}{R_{\text{out}}}\right) \\
        C_1 & = \frac{1}{2 \pi \cdot 868 \cdot 10^6} \cdot \left(\frac{15.27}{50}\right) \cdot \left(1 + 1.508^2 - \frac{1 + 1.508}{15.27}\right) \approx 5.53 \, \text{pF} \\
    \end{align}

    Filtr pasmowoprzepustowy -- wartości C2 oraz L2:
    \begin{align}
        C_2 & = \frac{1}{2 \pi H_2^2 \cdot L_2} \\
        H_2 & = 2 \cdot 868 \cdot 10^6 = 1.736 \, \text{GHz} \\
        C_2 & = \frac{1}{2 \pi \cdot (1.736 \cdot 10^9)^2 \cdot 3.34 \cdot 10^{-9}} \approx 2.52 \, \text{pF} \\
        L_2 & = \frac{3}{4} \cdot L_1 = \frac{3}{4} \cdot 4.45 \, \text{nH} = 3.34 \, \text{nH}
    \end{align}

    Filtr dolnoprzepustowy:
    \begin{align}
             L~  & = \frac{50}{2 \pi f} \\
             L~  & = \frac{50}{2 \pi \cdot 868 \cdot 10^6} \approx 9.18 \, \text{nH} \\
        C' = C'' & = \frac{0.95}{50 \cdot 2 \pi f}\\
        C' = C'' & = \frac{0.95}{50 \cdot 2 \pi \cdot 868 \cdot 10^6} \approx 3.47 \, \text{pF}
    \end{align}

    Na bazie obliczonych wartości oraz wzorując się na przykładowym zestawieniu BOM (ang. \textit{bill of materials}), przedstawionym w~dokumentacji \cite[48]{ST_AN5457}, dobrane
    zostały rzeczywiste elementy. Wszystkie kondensatory oraz cewki pochodzą od jednego producenta -- Murata. Cewki  z~seri \textit{LQW} posiadają wysokie poziomy dobroci \textit{Q} częstotliwośi
    rezonansowej drgań; kondensatory charakteryzują się niskim ESR (ang. \textit{equivalent series resistance}), dielektrykiem CG0(NP0) oraz oba rodzaje elementów są w~obudowie 0402, co
    umożliwia zapewnienie dopasowania impedancji w~większym stopniu, gdyż szerokość ścieżek jest zbliżona do wymiarów padów cewek i~kondensatorów. STMicroelectronics zaleca użycie,
    szczególnie w~przypadku nowych projektów, elementów precyzyjnych -- o~niskim zakresie tolerancji, gdyż w~przeciwnym razie efekty pasożytnicze w~połączeniu z~rozrzutem parametrów
    poszczególnych części składowych mogą znacząco pogorszyć finalny efekt pracy -- dopasowanie układu. W~procesie doboru elementów wykorzystane zostało 
    \href{https://ds.murata.co.jp/simsurfing/index.html?lcid=en-us&md5=12458f7b5acc116a5fd4de3b5daf5105}{internetowe narzędzie  \textbf{SimSurfing}}, za pomocą którego można
    obejrzeć charakterystyki konkretnych komponentów -- przykładowo przebieg modułu impedancji kondensatora w~funkcji częstotliwości, co pozwala w~szerszym zakresie kontrolować ich parametry.
    
    Każdy element wykorzystany w~projekcie został dodany do specjalnie do tego przeznaczonej biblioteki elementów. Elementy były zakupione na stronie dystrybutora \textit{Mouser}, skąd też były
    pobrane symbole, \textit{footprinty} i~modele 3D. 

    \begin{figure}[H]
        \centering \includegraphics[width=1\textwidth]{simsurfing.png}
        \caption{SimSurfing -- przykładowy wykres i~dane dla wybranego elementu}
        \label{fig:simsurfing}
    \end{figure}
    
    Ostatecznia postać toru nadawczego widoczna jest na ilustracji \ref{fig:sensor_rf}, a~toru odbiorczego na ilustracji \ref{fig:main_rf}. Układ może nadawać zarówno z~dużą, jak i~małą mocą sygnału -- służą do tego zwory R22 oraz 
    R23, dzięki którym wyprowadzany jest sygnał albo z RFO\_LP albo RFO\_HP. Droższą, lecz wygodniejszą w~użyciu alternatywą są przełączniki -- tak zwane \textit{RF-switche},
    które odpowiednio wysterowane mogłyby kontrolować przełączanie ścieżek. Postanowiono na maksymalną oszczędność, dlatego też w~przypadku zmiany mocy sygnału należałoby zmienić część elementów 
    znajdujących się w~torze RF. Celem zapewnienia odpowiedniej elastyczności dołożony został jeszcze jeden stopień dopasowująco--filtrujący (cewka L6, kondensator C26). Kondensator C27 umożliwia
    ewentualną blokadę składowej stałej.
    
    \begin{figure}[H]
        \centering \includegraphics[width=1\textwidth]{sensor_rf.png}
        \caption{Moduł pomiarowy: tor radiowy -- nadawczy}
        \label{fig:sensor_rf}
    \end{figure}
    
    \begin{figure}[H]
        \centering \includegraphics[width=1\textwidth]{main_rf.png}
        \caption{Stacja bazowa: tor radiowy -- odbiorczy}
        \label{fig:main_rf}
    \end{figure}

\section{Obwody drukowane}
\noindent Na etapie projektowania płytek PCB nacisk położono na uniknięcie podstawowych błędów, prostotę oraz odseparowanie torów radiowych, choć nie było to konieczne, gdyż układ składa się głównie
z części cyfrowej. Stąd też, między innymi niewielka ilość ścieżek prowadzonych na warstwie innej niż wierzchnia płytki. Na samym początku znaleziono producenta płytek PCB, który był w~stanie za~niewielką cenę
oraz w~możliwie krótkim czasie dostarczyć prototypowe egzemplarze. Wybór padł na \textit{JLCPCB} -- chińskiego dostawcę, który oferuje okresowo wysokie zniżki, w~szczególności dla nowych klientów. 
JLCPCB dostarcza klientom zestaw reguł projektowych, których należy przestrzegać, ponieważ wynikają one z~możliwości technologicznych parku maszynowego. Zestaw reguł projektowych -- (\textit{rules}) został wprowadzony
do programu Altium , który przy odpowiedniej konfiguracji automatycznie koryguje i~ostrzega w~przypadku ich naruszenia. Dodatkowo wykorzystana została możliwość przeprowadzenia
znacznie głębszej analizy, również w~oparciu o~wprowadzone wcześniej ograniczenia~-- tak zwana analiza (\textit{DRC}) (ang. \textit{design rule check}). Drugim krokiem było 
zadecydowanie o~układzie warstw w~projekcie. Wytwórca ma możliwość wykonania PCB zarówno jednowarstowych jak i~nawet 32~-warstwowych. Płytki posiadające tak znaczną liczbę warstw dają możliwość
bardzo dużej gęstości połączeń, minimalizują problem integralności sygnałowej oraz przesłuchów, więc są drogie. W~przypadku tego projektu zdecydowano się na cztery warstwy, dzięki czemu możliwa była
opcja kontroli impedancji, bez której dopasowanie toru radiowego byłoby bardzo trudne. Oprócz wyboru ilości warstw, należało wybrać również odpowiedni wariant grubości
poszczególnych warstw. JLCPCB oferuje czternaście standardowych opcji. Wybrany układ warstw -- \textit{JLC0416H-3313}, dla którego grubość ścieżki o~impedancji charakterystycznej 50 ohm była zbliżona
do gabarytów padów elementów w~obudowie 0402, które zostały wybrane do toru nadawczego i~odbiorczego.

Po konfiguracji programu przystąpiono do rozmieszczania elementów na planie prostokąta -- w~bliskiej odległości elementy z~tych samych sekcji. Proces projektowania obu płytek był bardzo podobny -- na początku
ułożona została sekcja zasilania, mikrokontroler oraz kondensatory filtrujące w~pobliżu padów. Następnie część radiowa -- położona zdecydowanie dalej od pozostałych elementów z~dodatkowym zabezpieczeniem
w postaci osłony z~przelotek (ang. \textit{via-shielding}). \href{https://www.altium.com/documentation/altium-designer/via-stitching-via-shielding-pcb?srsltid=AfmBOop3OkNDzUuFfS844VtpEmkzZ8sMglQ_hj78dT1XZrPT4O3YZGLM#including-shielding-copper-with-the-stitching}
{Praktykę taką zaleca między innymi dystrybutor samego oprogramowania.} Złącza zostały umieszone na krawędziach. Szerokość ścieżek RF została obliczona przez wbudowany w~program kalkulator, co pozwoliło
na kontrolę impedancji. Po rozmieszczeniu najważniejszych elementów położone zostały pozostałe, mniej istotne komponenty.

Szerokość ścieżek w~projekcie:
\begin{itemize}
    \item zasilania: 15--20 mils;
    \item RF: 6,665 mils;
    \item innych: 10 mils.
\end{itemize}

Parametry przelotek:
\begin{itemize}
    \item średnica otworu: 0,3 mm;
    \item średnica metalizacji (całkowita): 0,45 mm;
    \item przykryte soldermaską (ang. \textit{tenting}).
\end{itemize}


\subsection{Stacja bazowa}
\noindent   
    \begin{figure}[H]
        \centering \includegraphics[width=0.9\textwidth]{main_2d.png}
        \caption{Stacja bazowa --  widok 2D}
        \label{fig:main_2d}
    \end{figure}
    \begin{figure}[H]
        \centering \includegraphics[width=0.9\textwidth]{main_pcb_3d.png}
        \caption{Stacja bazowa --  widok 3D}
        \label{fig:main_pcb_3d}
    \end{figure}
\subsection{Moduł pomiarowy}
\noindent
    \begin{figure}[H]
        \centering \includegraphics[width=0.85\textwidth]{sensor_2d.png}
        \caption{Moduł pomiarowy --  widok 2D PCB}
        \label{fig:sensor_2d}
    \end{figure}
    \begin{figure}[H]
        \centering \includegraphics[width=0.85\textwidth]{sensor_pcb_3d.png}
        \caption{Moduł pomiarowy --  widok 3D PCB -- góra}
        \label{fig:sensor_pcb_3d}
    \end{figure}
    \begin{figure}[H]
        \centering \includegraphics[width=0.85\textwidth]{sensor_pcb_3d_bot.png}
        \caption{Moduł pomiarowy --  widok 3D PCB -- dół}
        \label{fig:sensor_pcb_3d_bot}
    \end{figure}

\section{Obliczenia i~symulacje}
    \subsection{Stabilizator modułu pomiarowego}
        \noindent W~przypadku modułu pomiarowego najważniejszą kwestią był długi czas życia na baterii, który można wydłużyć, maksymalizując sprawność układu. Dlatego też w~przypadku stopnia 
        zasilającego były brane pod uwagę przetwornice, zostały jednak odrzucone, ponieważ ich sprawność gwałtownie maleje wraz ze spadkiem prądu wyjściowego -- szczególnie jeśli prąd wyjściowy jest rzędu
        mikroamperów, tak jak na przykład w~trybie głębokiego uśpienia, w~którym układ pozostawałby większość czasu. Wejściowe napięcie modułu to 5 V, wyjściowe 3,3 V -- przy tak niewielkiej
        różnicy napięć zdecydowano się na stabilizator LDO (ang. \textit{low dropout regulator}). Dzięki LDO możliwe jest działanie przy małej różnicy napięć wejściowego i~wyjściowego, co przy
        pracy na baterii może wydłużyć czas pracy. Parametry elektryczne wybranego modelu stabilizatora -- \textit{ADP123AUJZ--R7}:

        \begin{itemize}
            \item regulowane napięcie wejściowe: od 2,3 V do 5,5 V;
            \item prąd wyjściowy: do 350 mA;
            \item dropout: do 150 mV;
            \item prąd zasilania: $I_{\mathrm{GND}}$ do $240\ \mu\text{A}$.
        \end{itemize}
        
        Na bazie dokumentacji \cite{ADP123_DS} obliczono wartości dzielnika rezystancyjnego ustalającego napięcie wyjściowe a~także maksymalą temperaturę złącza w~trakcie pracy. Szacunkowe wartości
        prądu pobierane przez urządzenie zostały przedstawione w~tabeli \ref{tab:sensor module current}.

        Napięcie wyjściowe przetwornicy:
        \begin{align}
            \label{eq:ADP resistor eq}
            V_{OUT} & = 0,5\text{ V} \cdot \left(1+\frac{R_1}{R_2}\right)+I_{BIAS} \cdot R_1,
        \end{align}
        gdzie zalecana wartość dla R1 to mniej niż 200k$\Omega$ ze względu na pobierany przez terminal ADJ niezerowy prąd $I_\text{BIAS}$. Metodą iteracyjną zostały wyznaczone wartości $R_1 = 165\ \mathrm{k\Omega}$ oraz
        $R_2 = 29,4\ \mathrm{k\Omega}$, które teoretycznie zapewniają odchylenie od napięcia 3,3 V na poziomie 10 mV. $I_\text{BIAS}$ jest pomijalnie mały w~obliczeniach \cite[11]{ADP123_DS} -- jego wkład wynosi tylko 
        0,3 \% przy założeniu temperatury pokojowej. 
        Podstawiając znalezione wartości do wzoru \ref{eq:ADP resistor eq}:
        \begin{align}
            \label{eq:ADP resistor}
            V_{OUT} & = 0,5\text{ V} \cdot \left(1+\frac{165\text{ k$\Omega$}}{29,4\text{ k$\Omega$}}\right) = 3,306\text{ V}
        \end{align}

        Temperatura złącza:
        \begin{align}
            \label{eq:thermal_ADP}
            T_J & = T_A + {[(V_\text{IN}-V_\text{OUT}) \cdot I_\text{LOAD}] \cdot \varTheta_\text{JA}},
        \end{align}
        gdzie:
        \begin{itemize}
            \item $T_J$ -- maksymalna temperatura złącza w~$^\circ$C,
            \item $T_A$ -- temperatura otoczenia w~$^\circ$C,
            \item $V_\text{IN}$ -- napięcie wejściowe w~V,
            \item $V_\text{OUT}$ -- napięcie wyjściowe w~V,
            \item $I_\text{LOAD}$ -- prąd obciążenia w~A,
            \item $\varTheta_\text{JA}$ -- współczynnik przewodnictwa termicznego złącze--otoczenie w~$^\circ$C/W.
        \end{itemize}
        W~najgorszym przypadku temperatura nie może przekroczyć 125$^\circ$C, to jest kiedy temperatura otoczenia wynosi 40$^\circ$C, urządzenie pobiera maksymalny prąd oraz występuje największy
        współczynnik przewodnictwa termicznego, czyli dla wąskich ścieżek. Po podstawieniu do wzoru (\ref{eq:thermal_ADP}):
        \begin{align}
            T_J & = 40^\circ\text{C}  + [(5,0\text{ V}-3,3\text{ V}) \cdot 0,148 \text{ A}] \cdot 170^\circ\text{C/W}=82,772^\circ\text{C}\cong 82.8^\circ\text{C}< 125^\circ\text{C}
        \end{align}
        Wartości prądów i~napięć oraz temperatura złącza mieszczą się w~dopuszczalnym zakresie.

        W~tabelach \ref{tab:sensor module current} oraz \ref{tab:main module current} znajdują się zaokrąglone w~górę wartości masymalnego poboru prądu
        (w przypadku mikrokontrolera wszystkie peryferia włączone, taktowane najszybszym sygnałem zegarowym, maksymalna moc transmisji) na bazie \cite[65,78]{ST_DS13105} oraz dokumentacji 
        poszczególnych komponentów:

        \begin{table}[H] \centering
            \caption{Szacunkowy pobór prądu -- moduł pomiarowy}
            \label{tab:sensor module current}
            \begin{tabular}{l c}
                \hline\hline    
                \multicolumn{1}{c}{\textbf{Układ}}&\multicolumn{1}{r}{\textbf{Prąd [mA]}}\\\hline
                STM32WLE5C8U6      &                                \\  
                --rdzeń            & 15,0                           \\
                --radio (Tx)       & 110,0                          \\ 
                TCXO               & 5,0                            \\ 
                diody LED          & 10,0                           \\
                sensor             & 8,0                            \\ \hline
                \multicolumn{1}{r}{\textbf{Suma}}& \textbf{148 mA}  \\ \hline
                ADP123 ($I_{LOAD}$ = 300 mA)     & 0,240            \\
                \hline\hline
            \end{tabular}
        \end{table}     

        \begin{table}[H] \centering
            \caption{Szacunkowy pobór prądu -- stacja bazowa}
            \label{tab:main module current}
            \begin{tabular}{l c}
                \hline\hline    
                \multicolumn{1}{c}{\textbf{Układ}}&\multicolumn{1}{r}{\textbf{Prąd [mA]}}\\\hline
                STM32WLE5C8U6               &                               \\  
                --rdzeń                     & 15,0                          \\ 
                --radio (Rx boosted, SMPS)  & 11,0                          \\ 
                TCXO                        & 4,0                           \\ 
                diody LED                   & 10,0                          \\
                wyświetlacz LCD             & 100,0                         \\
                brzęczyk                    & 90,0                          \\ \hline
                \multicolumn{1}{r}{\textbf{Suma}}& \textbf{230 mA}          \\
                \hline\hline
            \end{tabular}
        \end{table}

        \begin{table}[H] \centering
            \caption{Szacunkowy pobór prądu w~trybie oszczędzania energii -- moduł pomiarowy}
            \label{tab:sensor module current LP}
            \begin{tabular}{l c}
                \hline\hline    
                \multicolumn{1}{c}{\textbf{Układ}}&\multicolumn{1}{r}{\textbf{Prąd [$\mu$A]}}\\\hline
                STM32WLE5C8U6                                                       &                  \\ 
                --rdzeń (tryb \textit{standby})                                     & 0,36             \\
                --radio (tryb \textit{sleep})                                       & 0,15             \\ 
                ADP123 (bez obciążenia)                                             & 45,0             \\ \hline
                \multicolumn{1}{r}{\textbf{Suma}}& \textbf{45,51 $\mu$A}    \\ 
                \hline\hline
            \end{tabular}
        \end{table}
        
    \subsection{Moduł pomiarowy -- czas życia przy zasilaniu bateryjnym}
        \noindent Szacowany średni pobór prądu modułu pomiarowego obliczony został przy założeniu, że urządzenie będzie dokonywało 10 pomiarów w~ciągu doby. Czas pracy urządzenia w~kolejnych fazach:
        \begin{table}[H] \centering
            \caption{Średni czas pracy urządzenia w~ciągu doby -- 10 cykli}
            \label{tab:battery life expectancy}
            \begin{tabular}{l c c}
                \hline\hline 
                \multicolumn{1}{c}{\textbf{Faza}}&\multicolumn{1}{c}{\textbf{Średni prąd [mA]}}&\multicolumn{1}{c}{\textbf{Czas trwania fazy [s]}} \\\hline
                \textbf{Sen}                                                 & 0,046  &                 ---                         \\  
                \textbf{Pomiar}                                              & 38,0   &                  3                          \\ 
                \textbf{Transmisja}                                          & 138,0  &                  5                          \\
                \hline\hline
            \end{tabular}
        \end{table}
        Na tej podstawie średni pobór prądu w~ciągu doby:
        \begin{align}
            \label{eq:mean current eq}
            I_{avg} & = \frac{1}{T} \left( I_{sleep} \cdot t_{sleep} + I_{measure} \cdot  t_{measure} + I_{transmit} \cdot  t_{transmit} \right)
        \end{align}
        gdzie:
        \begin{itemize}
            \item $I_{\text{avg}}$ -- średni prąd w~ciągu doby w~mA;
            \item $t_{\text{sleep}}$, $t_{\text{measure}}$, $t_{\text{transmit}}$ -- średni prąd w~poszczególnych fazach w~mA;
            \item $T$ -- czas, za który prąd jest uśredniany w~sekundach;
            \item $P_{\text{sleep}}$, $P_{\text{measure}}$, $P_{\text{transmit}}$ -- czasy poszczególnych faz w~sekundach.
        \end{itemize}
        Po podstawieniu:
        \begin{align}
            \label{eq:mean current}
            I_{avg} & = \frac{1}{24\cdot3600 \text{ s}} \left({0,046 \text{ mA}  \cdot 86320 \text{ s} + 38,0 \text{ mA}  \cdot 3 \text{ s} \cdot 10 + 138,0 \text{ mA}  \cdot 5 \text{ s} \cdot 10} \right)
            = 0,139 \text{ mA}
        \end{align}

        Na bazie średniego poboru prądu oszacowana została długość pracy urządzenia przy zasilaniu bateryjnym. Dodatkowe założenia to:
        \begin{itemize}
            \item urządzenie zasilane przez 3 ogniwa 1,5 V  1500 mAh(Aa Duracell Lr6/Mn1500);
            \item płaska charakterystyka prąd-napięcie.
        \end{itemize}

        \begin{align}
            \label{eq:life_expectancy}
            T~& = \frac{C_{bat}}{I_{śr}}=\frac{3 \cdot 1500 \text{ mAh}}{0,139\text{ mA} \cdot 24\text{ h}} = 1348,8 \simeq ~ 1349\text{ dni} = 3,7\text{ lat}
        \end{align}

    \subsection{Moduł pomiarowy -- dopasowanie toru radiowego}
        \noindent Po wstępnym dobraniu elementów toru nadawczego przeprowadzono niezbędne symulacje pozwalające ocenić stopień dopasowania. W~symulacji uwzględnione zostały modele pojemności i~
        indukcyjności pasożytniczych, których wartości zostały oszacowane na bazie długości odcinków linii między kolejnymi elementami oraz wskazaniami programu Altium. Na ilustracji \ref{fig:parasitics}. widoczne
        są indukcyjność i~pojemność linii mikropaskowej na jednostkę długości, zgodne z~przyjętym stosem warstw PCB.
        \begin{figure}[H]
            \centering \includegraphics[width=0.6\textwidth]{parasitics.png}
            \caption{Pojemności i~indukcyjności pasożytnicze na płytce PCB}
            \label{fig:parasitics}
        \end{figure}
        Do obliczeń wykorzystany został arkusz kalkulacyjny Excel. Na ilustracji \ref{fig:parasitics sensor unit} oznaczone zostały kolejne ścieżki toru. Na bazie ich długości, która jest
        podawana prze program Altium, możliwe było przeprowadzenie kalkulacji. Wyniki zebrane zostały w~tabeli \ref{tab:parasitics lp hp}.

        \begin{figure}[H]
            \centering \includegraphics[width=1\textwidth]{parasitics_sensor_unit.png}
            \caption{Fragment modułu pomiarowego wraz ze ścieżkami uwzględnionymi w~obliczeniach pasożytniczych pojemności i~indukcyjności}
            \label{fig:parasitics sensor unit}
        \end{figure}
        \begin{table}[H] \centering
            \caption{Wyniki obliczeń pojemności i~indukcyjności pasożytniczych}
            \label{tab:parasitics lp hp}
            \begin{tabular}{c c c c}
                \hline\hline 
                \multicolumn{1}{c}{\textbf{Nr linii}} & \multicolumn{1}{c}{\textbf{L [mm]}} & \multicolumn{1}{c}{\textbf{L [nH]}} & \multicolumn{1}{c}{\textbf{C [pF]}} \\\hline
                1. & 3.993 & 1.18 & 0.47 \\  
                2. & 3.615 & 1.07 & 0.43 \\ 
                3. & 1.801 & 0.53 & 0.21 \\ 
                4. & 1.617 & 0.48 & 0.19 \\ 
                5. & 1.256 & 0.37 & 0.15 \\ 
                6. & 1.788 & 0.53 & 0.21 \\ 
                7. & 1.804 & 0.53 & 0.21 \\ 
                8. & 4.225 & 1.25 & 0.50 \\ 
                9. & 4.549 & 1.35 & 0.54 \\  
                10. & 3.615 & 1.07 & 0.43 \\ 
                11. & 2.132 & 0.63 & 0.25 \\ 
                12. & 1.084 & 0.32 & 0.13 \\ 
                7+8\textsuperscript{a} & 9.906 & 2.93 & 1.17 \\ 
                \hline\hline
            \end{tabular}
            \vspace{0.1cm}
            \begin{center}
                \footnotesize{\textsuperscript{a} Oznaczenie ,,7+8'' reprezentuje linię od kondensatora C25 do gniazda SMA.}
            \end{center}
        \end{table}
        Podobne rozważania zostały przeprowadzone dla stacji bazowej. Wyniki znajdują się w~tabeli \ref{tab:parasitics_main} a~zaznaczone ścieżki na \ref{fig:parasitics_main_unit}.
        \begin{figure}[H]
            \centering \includegraphics[width=1\textwidth]{parasitics_main_unit.png}
            \caption{Fragment PCB stacji bazowej wraz ze ścieżkami uwzględnionymi w~obliczeniach \\ pasożytniczych pojemności i~indukcyjności}
            \label{fig:parasitics_main_unit}
        \end{figure}
        \begin{table}[H] \centering
            \caption{Wyniki obliczeń pojemności i~indukcyjności pasożytniczych dla stacji bazowej}
            \label{tab:parasitics_main}
            \begin{tabular}{c c c c}
                \hline\hline 
                \multicolumn{1}{c}{\textbf{Nr linii}} & \multicolumn{1}{c}{\textbf{L [mm]}} & \multicolumn{1}{c}{\textbf{L [nH]}} & \multicolumn{1}{c}{\textbf{C [pF]}} \\\hline
                1. & 5.906 & 1.75 & 0.70 \\  
                2. & 3.769 & 1.12 & 0.45 \\ 
                3. & 3.667 & 1.09 & 0.43 \\ 
                4. & 4.722 & 1.40 & 0.56 \\ 
                5. & 6.078 & 1.80 & 0.72 \\ 
                6. & 3.769 & 1.12 & 0.45 \\ 
                7. & 2.618 & 0.78 & 0.31 \\ 
                \hline\hline
            \end{tabular}
        \end{table}
        Na podstawie zebranych danych możliwe było przeprowadzenie symulacji. Na parach ilustracji \ref{fig:symulacja_lp_par_schemat}. i~\ref{fig:symulacja_lp_par_wyniki}., 
        \ref{fig:symulacja_hp_par_schemat}. i~\ref{fig:symulacja_hp_par_wyniki}. oraz \ref{fig:symulacja_rx_par_schemat}. i~\ref{fig:symulacja_rx_par_wyniki}. przedstawione zostały obwody transmisyjny LP, HP oraz odbiorczy stacji bazowej wraz z~odpowiadającymi
        im wynikami symulacji.
        \begin{figure}[H]
            \centering \includegraphics[width=1\textwidth]{symulacja_lp_par_schemat.png}
            \caption{Obwód transmisyjny LP modułu pomiarowego wykorzystany w~symulacji}
            \label{fig:symulacja_lp_par_schemat}
        \end{figure}

        \begin{figure}[H]
            \centering \includegraphics[width=1\textwidth]{symulacja_lp_par_wyniki.png}
            \caption{Wyniki symulacji obwodu transmisyjnego LP}
            \label{fig:symulacja_lp_par_wyniki}
        \end{figure}

        \begin{figure}[H]
            \centering \includegraphics[width=1\textwidth]{symulacja_hp_par_schemat.png}
            \caption{Obwód transmisyjny LP modułu pomiarowego wykorzystany w~symulacji}
            \label{fig:symulacja_hp_par_schemat}
        \end{figure}

        \begin{figure}[H]
            \centering \includegraphics[width=1\textwidth]{symulacja_hp_par_wyniki.png}
            \caption{Wyniki symulacji obwodu transmisyjnego LP}
            \label{fig:symulacja_hp_par_wyniki}
        \end{figure}

        \begin{figure}[H]
            \centering \includegraphics[width=1\textwidth]{symulacja_rx_par_schemat.png}
            \caption{Obwód odbiorczy stacji bazowej wykorzystany w~symulacji}
            \label{fig:symulacja_rx_par_schemat}
        \end{figure}

        \begin{figure}[H]
            \centering \includegraphics[width=1\textwidth]{symulacja_rx_par_wyniki.png}
            \caption{Wyniki symulacji obwodu odbiorczego stacji bazowej}
            \label{fig:symulacja_rx_par_wyniki}
        \end{figure}

        \begin{table}[H] \centering
            \caption{Wyniki symulacji}
            \label{tab:parasitics results}
            \begin{tabular}{c c c}
                \hline\hline 
                \multicolumn{3}{c}{\textbf{\hspace{40pt}\uline{Moduł pomiarowy}}}           \\
                                            & \textbf{LP}           & \textbf{HP}           \\ \hline
                S21 [j.u.]                  &  -0,713+j0,70         & -0,682+j0,73          \\
                |S21|                       &  0,992                & 0,999                 \\
                P$_\text{TX}$ [\%]\textsuperscript{a}      &   99,8                &   99,8 \\
                S11 [dB]                    &   -26,9               &   -27,5               \\ \hline
                \multicolumn{3}{c}{\hspace{40pt}\textbf{Stacja bazowa}}                     \\ \hline
                S31 [j.u.]                  &  \multicolumn{2}{c}{0,066-j0,887}             \\
                |S31|                       &  \multicolumn{2}{c}{0,890}                    \\
                S21 [dB]                    &  \multicolumn{2}{c}{-24,6}                    \\ 
                P$_\text{TX}$ [\%] &  \multicolumn{2}{c}{79,1}                              \\
                \hline\hline
            \end{tabular}
            \vspace{0.1cm}
            \begin{center}
                \footnotesize{\textsuperscript{a} Procentowa ilość transmitowanego sygnału między wrotami wejściowymi i~wyjściowymi.}
            \end{center}
        \end{table}
        Na podstawie zebranych danych można stwierdzić, że obwody transmisyjne mają dobry stopień dopasowania, ponieważ na wrotach anteny odbija się tylko 1\% mocy. Obwód stacji bazowej
        jest dopasowany znacznie gorzej, gdyż z~anteny do mikrokontrolera dociera jedynie 79\% sygnału. Charakterystyki wszystkich trzech torów są bardzo wąskopasmowe, więc każdy 
        dodatkowy, nieuwzględniony element pasożytniczy, jest w~stanie ją mocno zmienić. Dlatego też poprawianie dopasowania obwodów powinno mieć miejsce dopiero po zlutowaniu układów. 
        Symulacja miała za zadanie pokazać, czy obliczenia wartości elementów zostały wykonane poprawnie. Przeprowadzone obliczenia są obarczone niedokładnością i~powinny być traktowane
        jedynie jako punkt odniesienia. Nie zostały uwzględnione pojemności padów, pojemności ścieżek do płaszczyzny masy na warstwie \textit{TOP}, a~także dodatkowe pojemności i~
        indukcyjności pasożytnicze samych elementów, ponieważ dokładne dane nie zostały ukazane w~ich notach katalogowych, a~charakterystyki, choć dostępne w~SimSurfing, nie były wystarczające
        do ekstrakcji wszystkich niezbędnych parametrów, ponieważ były nieliniowe i~mocno odbiegały od teoretycznych charakterystyk kondensatorów i~cewek indukcyjnych. 

\clearpage\chapter{Projekt obudowy}
\noindent Obudowa została zaprojektowana w~programie Inventor oraz wydrukowana na drukarce 3D Ender-3 S1 Pro. Proces tworzenia obudowy polegał na wyeksportowaniu z~programu Altium 
modeli płytek PCB w~formacie .step oraz dopasowaniu do nich tworzonej geometrii. Model czujnika ultradźwiękowego został pobrany z~Internetu. Rezultat prac widoczny jest na poniższej 
ilustacji \ref{fig:obudowy}.

\begin{figure}[!h]
    \centering
    \begin{subfigure}[b]{0.45\textwidth}
        \centering
        \includegraphics[width=\textwidth]{sensor_3dd.png}
    \end{subfigure}
    \hspace{0.05\textwidth} % odstęp między podrysunkami
    \begin{subfigure}[b]{0.45\textwidth}
        \centering
        \includegraphics[width=\textwidth]{main_3dd.png}
    \end{subfigure}
    \caption{Projekt obudowy. Po lewej obudowa czujnika. Po prawej obudowa stacji bazowej}
    \label{fig:obudowy}
\end{figure}

\part{Realizacja prototypu i~weryfikacja} 
\clearpage\chapter{Montaż}
\noindent Po dostarczeniu elementów oraz płytek PCB (nowe płytki PCB na zdjęciach \ref{fig:before soldering home} oraz \ref{fig:before soldering sensor}) przystąpiono do montażu urządzeń. Z~uwagi
na dużą ilość elementów w~obudowach SMD oraz dostępność półautomatycznego dyspensera pasty lutowniczej oraz pieca zdecydowano się na lutowanie rozpływowe (ang. \textit{reflow soldering}), które nie wymaga 
użycia topnika, zatem podłoże po zakończonym procesie jest czyste. Metoda ta eliminuje niebezpieczeństwo zdmuchnięcia elementów gorącym powietrzem (ang. \textit{hot-air}) oraz
upraszcza lutowanie małych, nieporęcznych elementów. Na samym początku położone zostały rezystory, kondensatory i~cewki indukcyjne,
a także koraliki ferrytowe. W~drugiej iteracji położone zostały większe elementy -- układy scalone, generatory TCXO oraz pozostałe elementy do~montażu powierzchniowego. Na samym końcu 
zostały ręcznie polutowane gniazda SMA oraz złącza szpilkowe. 

\begin{figure}[H]
    \centering \includegraphics[width=0.8\textwidth]{main_bf_soldering.JPG}
  \caption{Płytka PCB stacji bazowej przed montażem}
    \label{fig:before soldering home}
\end{figure}

\begin{figure}[H]
    \centering \includegraphics[width=0.8\textwidth]{sensor_bf_soldering.JPG}
  \caption{Płytka PCB modułu pomiarowego przed montażem}
    \label{fig:before soldering sensor}
\end{figure}

Na ilustracji \ref{fig:PCB_solder}. widoczne są dwie płytki PCB przed pierwszym lutowaniem. Pady obudów 0402 były na tyle małe, że dyspenser przy ustawieniu najmniejszej dostępnej kropli
pasty wciąż dozował jej zbyt wiele -- pasty było za dużo i~po trzydziestu minutach zaczynała się rozpływać na boki. Dodatkowo wytrącała się z~niej ciecz, co powodowało, że komponenty 
przestawały się trzymać podłoża. Z~tego względu co pewien czas pasta musiała zostać schłodzona w~lodówce, żeby odzyskała swoje pierwotne właściwości.
Z uwagi na wymienione problemy nakładanie pasty i~lutowanie w~piecu odbywało się naprzemian w~kilku cyklach. 

\begin{figure}[H]
  \centering \includegraphics[width=0.8\textwidth]{PCB_solder.JPG}
  \caption{Płytki przed lutowaniem w~piecu rozpływowym}
  \label{fig:PCB_solder}
\end{figure}

\clearpage
Na zdjęciu \ref{fig:main_zlut}. widoczny jest efekt prac nad modułem stacji bazowej.
\begin{figure}[H]
  \centering \includegraphics[width=0.8\textwidth]{main_zlut.JPG}
  \caption{Stacja bazowa po zlutowaniu}
  \label{fig:main_zlut}
\end{figure}

\clearpage\chapter{Pierwsze uruchomienie}
\noindent Po zakończonym montażu przystąpiono do uruchomienia urządzeń. Okazało się, że popełniono kilka błędów, w~skutek czego uszkodzono aż pięć układów STM32WL. 
Objawem był nienaturalnie wysoki pobór prądu obu urządzeń oraz nadmiernie wydzielane ciepło. Po pierwsze i~najważniejsze, okazało się, że w~wyniku przeoczenia został źle zaprojektowany fragment
sekcji zasilania wspólny dla obu urządzeń. Zgodnie z~tym, co widać na ilustracji \ref{fig:VDDPA_issue} wyprowadzenia VDDPA (3,3~V) oraz VDDRF1V55 (1,55~V), zostały połączone na stałe. Zgodnie z~dokumentacją
\cite[23]{ST_DS13105}, wyjście VDDRF1V55 dostarcza zasilanie dla modułu radiowego i~powinno być połączone z~VFBSMPS (wyjście LDO lub przetwornicy SMPS), jednak w~wyniku przylutowania rezystora R13, VDDRF1V55 
zostało zwarte do~zasilania 3,3~V. Wyeliminowano problem poprzez wylutowanie rezystora R13 na stałe i~połączenie VDDPA oraz VDDRF1V55 z~VFBSMPS przez R14. Niestety, w~wyniku tej operacji utracona została
możliwość nadawania sygnału radiowego z~maksymalną mocą +22 dBm (maksymalnie +14~dBm), gdyż w~tym celu VDDPA potrzebuje napięcia 3,3 V. Ilustracja \ref{fig:nagrzany_mcu} przedstawia zdjęcie wykonane kamerą
termowizyjną grzejącego się układu. Pobierany prąd z~zasilacza wynosił około 120~mA. Po drugie, dobrano zbyt małe rezystory ograniczające prądy diod, sugerując się jedynie mało dokładną charakterystyką
prąd--napięcie. Dodatkowo, jeden układ został uszkodzony wyładowaniem ESD.

\begin{figure}[htbp]
  \centering \includegraphics[width=0.5\textwidth]{nagrzany_mcu.png}
  \caption{PCB z~uszkodzonym mikrokontrolerem}
  \label{fig:nagrzany_mcu}
\end{figure}

\begin{figure}[htbp]
  \centering \includegraphics[width=0.7\textwidth]{VDDPA_issue.png}
  \caption{Błąd w~schemacie elektrycznym}
  \label{fig:VDDPA_issue}
\end{figure}

Kolejnym problematycznym elementem był generator TCXO. Próba zmierzenia sygnału sondą oscyloskopową w~trybie skalowania \textbf{$\times 1$} nie dawała oczekiwanych rezultatów. Sygnał, który zgodnie
z założeniami \cite{ECS_TCXO_DS} miał być sinusoidą o~amplitudzie minimum 0,8 V, miał około 20 mV i~co pewien czas ustawał. Błędnie przypuszczano, że generator został uszkodzony w~trakcie procesu
odlutowywania i~ponownego lutowania mikrokontrolera z~użyciem gorącego powietrza. W~efekcie TCXO było kilkukrotnie zmieniane. Finalnie okazało się, że problem leży w~naturze samego pomiaru -- konkretnie
w budowie sondy oscyloskopowej, która w~ustawieniu \textbf{$\times 1$} zwiększa pojemność widzianą z~poziomu wejścia oscyloskopu. TCXO jest w~stanie wysterować tylko około 10 pF, gdzie pojemność sondy 
prawdopodobnie zwielokrotniła pojemność widzianą z~poziomu generatora -- ten nie był w~stanie zapewnić odpowiedniego prądu i~zostawał wyłączony. W~rozdziale \ref{errors}. omówione zostały wszystkie 
problemy, które zostały zauważone na różnych etapach realizacji projektu. 

Pomimo trudności udało się z~powodzeniem uruchomić oba urządzenia i~przetestować działanie poszczególnych peryferiów -- w~tym wyświetlacza LCD, brzęczyków, czujnika ultradźwiękowego oraz TCXO. 

Następnym etapem były testy komunikacji bezprzewodowej. Niestety, pomimo wielu godzin spędzonych nad projektem, nie udało się doprowadzić do przesłania danych z~modułu pomiarowego do stacji bazowej. Na tym
etapie, mając na uwadze ilość zaistniałych problemów, zadecydowano o~konieczności stworzenia kolejnej rewizji -- pozbawionej problemów pierwszej wersji.

\clearpage\chapter{Popełnione błędy} \label{errors}
  \noindent W~trakcie uruchamiania urządzeń napotkano szereg problemów. Podczas projektowania schematów popełniono krytyczny błąd, który doprowadził do uszkodzenia kilku sztuk układu STM32WLE5C8U6,
  co w~efekcie opóźniło projekt (układy nie są dostępne w~lokalnej dystrybucji) i~zwiększyło nakład finansowy. W~wyniku przeoczenia VDDPA oraz VDDRF1V55 zostały na stałe połączone, więc próba zasilenia
  VDDPA z~napięcia 3,3 V niezbędnego do transmisji z~dużą mocą 21~dBm prowadziła do uszkodzenia terminalu VDRF1V55 i~całego radia. 
  Rozwiązaniem było montaż rezystorów w~odpowiedniej konfiguracji i~ograniczenie mocy w~układzie nadawczym do 14 dBm. 
  
  Dobrane zostały zbyt małe rezystory dla diod LED, co było wynikiem zbyt dużych uproszczeń i~doprowadziło do uszkodzenia GPIO -- w~notach katalogowych diod dołączone były jedynie mało dokładne
  charakterystyki prąd -- napięcie. Należało na tamtym etapie zmienić model elementu, rozważyć sterowanie kluczem tranzystorowym albo użyć bezpiecznych, większych rezystancji.
  
  Kolejnym problemem było to, że nie przewidziano kluczowania zasilania dla sensora ultradźwiękowego, co negatywnie wpłynęłoby na długość życia urządzenia zasilanego z~baterii (według szacunków przy stałym 
  poborze prądu rzędu $8,0\ \mathrm{mA}$ czas życia wyniósłby tylko około 20 dni). Rozwiązaniem jest zmiana konfiguracji jednego z~wyprowadzeń, które pierwotnie miało służyć do celów diagnostycznych interfejsu SPI
  (rys. \ref{fig:sensor_issue}), na \textit{push--pull} oraz zewnętrznego klucza tranzystorowego. 

  \begin{figure}[H]
    \centering \includegraphics[width=0.7\textwidth]{sensor_issue.png}
    \caption{Dolna strona płytki sensora i~wyprowadzenia diagnostyczne}
    \label{fig:sensor_issue}
  \end{figure}

  W~trakcie projektowania płytek PCB nie uwzględniono efektów pasożytniczych w~sekcjach ,,krytycznych". Przykładowo: na płytce stacji bazowej dodano dużą indukcyjność pasożytniczą w~torze RF,
  ponieważ kondensator C21 ma połączenie z~masą tylko jedną przelotką, co widoczne jest poniżej na rys. \ref{fig:masa_issue}.

  \begin{figure}[H]
    \centering \includegraphics[width=0.7\textwidth]{masa_issue.png}
    \caption{Dodatkowa indukcyjność w~torze RF}
    \label{fig:masa_issue}
  \end{figure}

  Nieoptymalne są również ścieżki prowadzące od wyjścia TCXO do STM32 (rys. \ref{fig:clock_issue}). Szerokie ścieżki dodają niepotrzebnie pasożytniczą pojemność. Można było także użyć elementów w~obudowach 0402, które mają mniejsze pady.

  \begin{figure}[H]
    \centering \includegraphics[width=0.45\textwidth]{clock_issue.png}
    \caption{Zbyt szerokie ścieżki TCXO}
    \label{fig:clock_issue}
  \end{figure}

  Dodatkowo zaniedbana została warstwa opisowa na obu płytkach, przez co konieczne jest częste sięganie po dokumentację, żeby sprawdzić, gdzie na złączach szpilkowych wyprowadzony został jaki sygnał. 
  Na płytkach brakuje też punktów pomiarowych oraz złączy, w~które można przykładowo wpiąć amperomierz, co uprościłoby przeprowadzenie pomiarów związanych z~zasilaniem.
  Niektóre elementy na płytce mogłyby zostać wymienione. Przykładem mogą być złącza szpilkowe, które służą raczej prototypowaniu. Nie ma także potrzeby używania rezystorów w~obudowach 0402.
  Elementy te są małe i~trudniejsze w~lutowaniu od komponentów w~obudowach 0603 czy 0805. Małe rezystory, kondensatory czy cewki powinny być używane tylko wtedy, kiedy zachodzi rzeczywista potrzeba.

%Komunikacja na początku była ustawiona programowo na 868,7 MHz, co zostało wykryte po pierwszym dopasowywaniu obwodu. 
%Przy trymowaniu obwodu RF jeden z~padów odpadł. Cewki lutowało się słabo i~w efekcie czasami nie stykały co bardzo wydłużyło czas trymowania obwodów.

\part{Rewizja druga} 
  \clearpage \chapter{Projekt}
    \noindent W~związku z~nieprawidłowym działaniem pierwszej wersji projektu oraz trudnościami w~znalezieniu przyczyny stworzono drugą, poprawioną wersję. Wprowadzono dwie kluczowe z~punktu 
    projektowego zmiany. Zamiast dwóch płytek PCB powstała jedna, co przełożyło się na znaczące obniżenie kosztów oraz skrócenie czasu wytworzenia, a~także lepsze dopracowanie układu. Drugą
    istotną różnicą było całkowicie odmienne podejście do projektu części radiowej -- tym razem wzorowano się na referencyjnym projekcie od STMicroelectronics opartym o~scalony balun IPD 
    (ang. \textit{integrated passive device}). Takie podejście ma szereg zalet \cite[65]{ST_AN5457} w~porównaniu z~budowaniem toru radiowego na elementach dyskretnych:

    \begin{itemize}
        \item prostsze dopasowanie impedancji;
        \item tańsze testowanie;
        \item brak problemu rozrzutu parametrów,
        \item mniejsza liczba elementów,
        \item zmiejszenie wymiarów PCB.
    \end{itemize}

    Nowy projekt umożliwił rozwiązanie wielu problemów, które miał stary układ. Wprowadzone poprawki to:
    \begin{itemize}
        \item eliminacja problemu bezpośredniego połączenia VDD i~VDDRF1V55;
        \item dodanie kluczowania zasilania sensora;
        \item dodanie czytelnej, odpowiedniej warstwy opisowej;
        \item dodanie punktów, rezystorów i~złączy pomiarowych;
        \item dodanie złącz i~diod do celów debugowania;
        \item optymalizacja szerokości ścieżek;
        \item wkorzystanie z~większych elementów (rozmiar 0603 zamiast 0402).
    \end{itemize}

    \section{Schematy elektryczne}
        \noindent Nowy tor radiowy to nie jedyna zmiana w~projekcie. Pomniejszym modyfikacjom uległa każda wyodrębniona sekcja. W~przypadku mikrokontrolera (rys. \ref{fig:rev2_mcu}) zostały dodane między innymi: 
        sterowanie przełącznikiem RF oraz zasilaniem sensora, przycisk oraz opcjonalny rezystor R13 w~torze TCXO. Wyprowadzony został również sygnał MCO, który umożliwia zweryfikowanie poprawności
        sygnału zegarowego.
        \begin{figure}[htbp]
            \centering \includegraphics[width=1\textwidth]{rev2_mcu.png}
            \caption{Rewizja druga -- mikrokontroler}
            \label{fig:rev2_mcu}
        \end{figure}

        Do sekcji zasilania (rys. \ref{fig:rev2_pwr}) dodano przycisk resetu, punkty testowe głównych napięć w~układzie, zwory oraz rezystory pomiarowe prądu. Ważną zmianą było
        połączenie na stałe VDDPA z~VDDRF1V55, co odpowiada konfiguracji niskomocowej modułu radiowego (14 dBm). Wybrano także mocniejszy stabilizator (ADP124 zamiast APD123) -- o~większym prądzie
        maksymalnym, tak aby możliwe było zasilenie bardziej prądożernego wyświetlacza LCD. %\textcolor{red}{\textbf{"prądożernego"}} wyświetlacza LCD.
        \begin{figure}[htbp]
            \centering \includegraphics[width=1\textwidth]{rev2_pwr.png}
            \caption{Rewizja druga -- sekcja zasilania}
            \label{fig:rev2_pwr}
        \end{figure}

        Tor radiowy (rys. \ref{fig:rev2_radio}) powstał na bazie układu \cite{ST_REFERENCE} z~materiałów pomocniczych dostarczonych przez producenta. Użyty został balun w~postaci układu IPD oraz przełącznik
        RF, który umożliwił dwustronną komunikację -- transmisję i~odbiór danych.
        \begin{figure}[htbp]
            \centering \includegraphics[width=1\textwidth]{rev2_radio.png}
            \caption{Rewizja druga -- tor radiowy}
            \label{fig:rev2_radio}
        \end{figure}

        Obok bezprzewodowej transmisji danych, urządzenie w~swoich założeniach miało być energooszczędne. Stąd też prawdopodobnie druga najważniejsza zmiana względem poprzedniej rewizji, czyli
        dodanie możliwości włączania i~wyłączania sensora, zrealizowane na kluczu tranzystorowym widocznym na rys. \ref{fig:rev2_sensor}. 
        \begin{figure}[htbp]
            \centering \includegraphics[width=0.5\textwidth]{rev2_sensor.png}
            \caption{Rewizja druga -- interfejs sensora}
            \label{fig:rev2_sensor}
        \end{figure}

        W~nowej odsłonie projektu nie zabrakło zabezpieczeń ESD -- każde złącze szpilkowe ma swoją diodę zabezpieczającą przed wyładowaniem elektrostatycznym.
        Przy wyborze i~umiejscowieniu elementów stosowano się do wskazówek zawartych w~\cite{ST_AN5612}.
        \begin{figure}[htbp]
            \centering \includegraphics[width=0.7\textwidth]{rev2_lcd_prog.png}
            \caption{Rewizja druga -- interfejs LCD i~programatora}
            \label{fig:rev2_lcd_prog}
        \end{figure}

        Urządzenia komunikujące się w~jakiejkolwiek sieci zazwyczaj mają unikalny adres, który je identyfikuje. Często jest to tzw. adres MAC (ang. \textit{Medium Access Control}),
        dzięki któremu możliwe jest określenie, kto jest nadawcą lub odbiorcą danych. W~projekcie posłużono się uproszczoną wersją, opartą o~konfigurację rezystorów na
        płytce PCB, co widoczne jest na \ref{fig:rev2_buzz_id} w~sekcji ID.
        \begin{figure}[htbp]
            \centering \includegraphics[width=1\textwidth]{rev2_buzz_id.png}
            \caption{Rewizja druga -- brzęczyk i~ID}
            \label{fig:rev2_buzz_id}
        \end{figure}


    \section{PCB}
        \noindent Projekt obwodu drukowanego oparto na \cite{ST_REFERENCE}. Na początku, w~identyczny sposób jak w~projekcie referencyjnym, zaprojektowano tor 
        radiowy i~następnie pozostałe elementy. Dużą uwagę poświęcono optymalizacji położenia kondensatorów filtrujących zasilanie. Uniwersalność i~większe komponenty
        (stacja bazowa i~moduł pomiarowy w~jednym) przyczyniły się do powiększenia gabarytów obwodu. Nie było to problemem, gdyż nie zostały narzucone żadne wymagania techniczne
        dotyczące wymiarów urządzenia. Gotowy projekt PCB jest na rys. \ref{fig:rev2_PCB}. Częściowo zlutowany jest na rys. \ref{fig:PCB_zlut}.

        \begin{figure}[htbp]
            \centering
                \begin{subfigure}[b]{0.46\textwidth}
                    \centering
                    \includegraphics[width=\textwidth]{rev2_front.png}
                \end{subfigure}
                \hspace{0.05\textwidth} % odstęp między podrysunkami
                \begin{subfigure}[b]{0.44\textwidth}
                    \centering
                    \includegraphics[width=\textwidth]{rev2_back.png}
                \end{subfigure}
            \caption{Rewizja druga -- gotowy projekt obwodu drukowanego}
            \label{fig:rev2_PCB}
        \end{figure}


        \begin{figure}[htbp]
            \centering \includegraphics[width=0.8\textwidth]{PCB_zlut.JPG}
            \caption{Rewizja druga -- częściowo zlutowany projekt}
            \label{fig:PCB_zlut}
        \end{figure}

\clearpage\chapter{Testy}
 
    \section{Testy komunikacji bezprzewodowej}
    \noindent Początkowo, bez żadnej optymalizacji, zasięg komunikacji bezprzewodowej wynosił około 10 metrów. Pierwszym krokiem było zbadanie dopasowania nowego toru radiowego do anteny.
    Zły poziom dopasowania obu układów poprawiono do akceptowalnego poziomu oraz upewniono się, że radia są skonfigurowane pod kątem maksymalnego zasięgu komunikacji.
    Na poniższych zdjęciach widoczne są wartości dopasowania impedancji zarówno przed, jak i~po trymowaniu dla obu układów (rys. \ref{fig:rev2_home_trymming} i~\ref{fig:rev2_sen_trymming}). 

      \begin{figure}[!h]
            \centering
                \begin{subfigure}[b]{0.45\textwidth}
                    \centering
                    \includegraphics[width=\textwidth]{rev2_home_start.JPG}
                \end{subfigure}
                \hspace{0.05\textwidth} % odstęp między podrysunkami
                \begin{subfigure}[b]{0.45\textwidth}
                    \centering
                    \includegraphics[width=\textwidth]{rev2_home_end.JPG}
                \end{subfigure}
            \caption{Rewizja druga -- poziom dopasowania stacji bazowej przed (po lewej) i~po trymowaniu (po prawej)}
            \label{fig:rev2_home_trymming}
        \end{figure}

        \begin{figure}[!h]
            \centering
                \begin{subfigure}[b]{0.45\textwidth}
                    \centering
                    \includegraphics[width=\textwidth]{rev2_sen_start.JPG}
                \end{subfigure}
                \hspace{0.05\textwidth} % odstęp między podrysunkami
                \begin{subfigure}[b]{0.45\textwidth}
                    \centering
                    \includegraphics[width=\textwidth]{rev2_sen_end.JPG}
                \end{subfigure}
            \caption{Rewizja druga -- poziom dopasowania modułu pomiarowego przed (po lewej) i~po trymowaniu (po prawej)}
            \label{fig:rev2_sen_trymming}
        \end{figure}

        \begin{table}[H] \centering
                \caption{Poziomy impedancji wejściowej przed i~po dopasowaniu obwodu}
                \label{tab:trymming}
                \begin{tabular}{r c c}
                    \hline\hline 
                    \multicolumn{1}{r}{\textbf{urządzenie}}&\multicolumn{1}{c}{\textbf{impedancja przed dopasowaniem [$\Omega$]}}&\multicolumn{1}{c}{\textbf{impedancja po dopasowaniu [$\Omega$]}} \\\hline
                    \textbf{moduł pomiarowy}                                                & 46,5 - j33,6  &               65,1 - j13,5                  \\  
                    \textbf{stacja bazowa}                                                  & 44,2 - j36,7  &               64,9 - j8,8                   \\ 
                    \hline\hline
                \end{tabular}
        \end{table}

        Na podstawie zebranych wyników przeprowadzono obliczenia parametrów takich jak SWR (ang. \textit{standing wave ratio}) oraz $\Gamma$ (współczynnik odbicia)--do łatwiejszego oszacowania poziomu dopasowania toru.
        Współczynnik odbicia $\Gamma$:

        \begin{align}
            \label{eq:gamma}
            \Gamma &= \frac{Z_L - Z_0}{Z_L + Z_0} = \frac{(46.5 - j33.6) - 50}{(46.5 - j33.6) + 50}  \\
            \Gamma &= \frac{-3.5 - j33.6}{96.5 - j33.6} = 0,075 - j0,322  
        \end{align}
        
        Stąd moduł wyrażenia:
        \begin{align}
            |\Gamma| &= \sqrt{(0,075)^2 + (0,322)^2} = 0.331 
        \end{align}
       
        Na tej podstawie SWR oraz procent mocy odbitej od anteny (moc strat):
        \begin{align}
            P_{\text{odbita}} &=  |\Gamma|^2 \cdot 100\% =  0.331^2 \cdot 100\% = 10,9\% \\
            SWR &= \frac{1 + |\Gamma|}{1 - |\Gamma|} = \frac{1 + 0.331}{1 - 0.331} = 1.99 
        \end{align}

        W~identyczny sposób obliczono parametry dopasowania we wszystkich pozostałych trzech przypadkach. Wyniki zebrano w~tabeli poniżej:
        \begin{table}[H] \centering
                \caption{Poziomy SWR, współczynnika odbicia i~procent mocy strat modułów przed i~po trymowaniu obwodu}
                \label{tab:trymming_params}
                \begin{tabular}{r c c c}
                    \hline\hline 
                    \multicolumn{1}{r}{\textbf{urządzenie}}&\multicolumn{1}{c}{\textbf{SWR}}&\multicolumn{1}{c}{\textbf{$|\Gamma|$}}&\multicolumn{1}{c}{\textbf{moc tracona [\%]}} \\\hline
                    \multicolumn{1}{r}{\textbf{przed dopasowaniem}}&\multicolumn{3}{c}{} \\ \hline
                    \textbf{moduł pomiarowy}                                        & 1,98      &                       0,33                                &  10,9\%       \\  
                    \textbf{stacja bazowa}                                          & 2,16      &                       0,37                                &  13,5\%       \\ \hline
                    \multicolumn{1}{r}{\textbf{po dopasowaniu}}&\multicolumn{3}{c}{} \\ \hline
                    \textbf{moduł pomiarowy}                                        & 1,42      &                       0,17                                &  3,1\%        \\  
                    \textbf{stacja bazowa}                                          & 1,35      &                       0,15                                &  2,3\%        \\ 
                    \hline\hline
                \end{tabular}
        \end{table}

        
        Kolejnym krokiem było sprawdzenie zasięgu komunikacji w~terenie. Pomiary zostały wykonane w~następujący sposób: stacja bazowa była trzymana przez jedną osobę, która odczytywała wyświetlane
        na ekranie wartości SNR, RSSI oraz flagi informujące o~stanie komunikacji. Druga osoba oddalała się z~odstępem 10 metrów wraz z~modułem pomiarowym -- aż do całkowitego zaniku komunikacji. Odległość
        między urządzeniami wyznaczana była przez aplikację zainstalowaną na telefonach obu osób biorących udział w~teście. Pierwotnie spodziewano się, że komunikacja będzie działała na znacznie
        większym dystansie, rzędu jednego lub dwóch kilometrów. Okazało się jednak, że zanika przy 200 metrach. Wartości SNR oraz RSSI zebrane w~trakcie testu przedstawiono na ilustracji.


        \begin{figure}[H]
            \centering \includegraphics[width=0.8\textwidth]{ch_ka_mocy.png}
            \caption{Badanie zasięgu urządzenia}
            \label{fig:ch_ka_mocy}
        \end{figure}
    
        Przyczyną był prawdopodobnie fakt przeprowadzania testów na zaszumionych obszarach miejskich oraz trzymanie urządzeń nisko nad ziemią. Przeprowadzono test w~skrajnie niekorzystnych warunkach, co oznacza
        że w~docelowych lokalizacjach, w~których przewiduje się eksploatację urządzeń, zasięg może być istotnie większy.

\clearpage\chapter{Wnioski}
\noindent 
    Przeprowadzone testy potwierdziły, że urządzenie spełnia założenia techniczne przyjęte na początkowym etapie pracy. Efekt można zobaczyć na rys. \ref{fig:final} i~\ref{fig:bazowa_oon}.
    Bezprzewodowy czujnik do kontroli poziomu w~zbiornikach asenizacyjnych został zrealizowany w~dwóch rewizjach. Urządzenia w~ramach pierwszej rewizji nie komunikowały się 
    ze sobą, prawdopodobnie z~powodu efektów pasożytniczych w~torze radiowym. Oprócz tego wystąpiły tam inne problemy, a~najpoważniejszym było to, że nie przewidziano kluczowania zasilania w~module sensora,
    który powinien być maksymalnie energooszczędny, co w~efekcie mogło doprowadzić do drastycznego skrócenia czasu życia na baterii, sprawiając, że urządzenie nie spełniałoby podstawowych założeń.
    
    Kolejna rewizja została lepiej przemyślana -- postawiono na sprawdzone rozwiązanie. Tor radiowy zaprojektowano wzorując się na przykładzie układu referencyjnego od STMicroelectronics.
    Zrezygnowano też z~dwóch bardzo podobnych układów na rzecz jednego, który z~powodzeniem mógłby realizować funkcję zarówno stacji bazowej, jak i~modułu pomiarowego w~zależności od tego, czy 
    założony zostałby ekran LCD, czy czujnik poziomu cieczy. Przy drugim podejściu skupiono się na minimalizacji popełnianych błędów i~rozwiązaniu problemów rewizji pierwszej, co zaowocowało działającym urządzeniem.
    
    Wnioski płynące z~realizacji pracy są następujące:
    \begin{enumerate}
        \item Należy poszukiwać sprawdzonych rozwiązań, ponieważ skraca to czas potrzebny na realizację projektu i~ryzyko na popełnienie błędów.
        \item Nie należy przyspieszać ani pomijać etapu sprawdzania projektu, gdyż jest on równie ważny jak etap projektowania -- w~przeciwnym razie cała praca może pójść na marne.
        \item Na etapie obmyślania koncepcji urządzenia trzeba myśleć również o~tym, jak dane urządzenie będzie testowane.
    \end{enumerate}
    Realizacja tego projektu podkreśla znaczenie starannego planowania oraz przemyślanego podejścia do~procesu projektowego. Pierwsza rewizja stanowi przykład, jak pośpiech i~brak dokładnej analizy
    mogą prowadzić do niepowodzenia, natomiast druga pokazuje, że wdrażanie sprawdzonych rozwiązań oraz uwzględnianie doświadczeń z~wcześniejszych etapów pozwala osiągnąć zamierzony, pozytywny efekt.



    \begin{figure}[H]
        \centering \includegraphics[width=0.8\textwidth]{final.JPG}
        \caption{Złożone stacja bazowa i~moduł pomiarowy}
        \label{fig:final}
    \end{figure}

       \begin{figure}[H]
        \centering \includegraphics[width=0.8\textwidth]{bazowa_oon.JPG}
        \caption{Stacja bazowa z~wyświetlaną grafiką}
        \label{fig:bazowa_oon}
    \end{figure}