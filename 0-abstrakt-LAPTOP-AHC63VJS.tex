\begin{pl_abstract}
    Współcześnie brakuje ekonomicznych, bezprzewodowych i~jednocześnie bezkontaktowych rozwiązań umożliwiających monitorowanie poziomu zapełnienia
    przydomowych zbiorników asenizacyjnych. W~niniejszej pracy przedstawiono projekt układu odpowiadającego na te potrzeby. Opracowane 
    rozwiązanie obejmuje dwa odrębne urządzenia: moduł pomiarowy wykorzystujący czujnik ultradźwiękowy do pomiaru poziomu cieczy oraz stację
    bazową; oba oparte o~mikrokontroler z~serii STM32WL z~wbudowanym interfejsem radiowym LoRa. Przeprowadzone testy potwierdziły poprawne działanie
    modułów, które stanowią fundament do stworzenia kompleksowej platformy monitorowania, wpisującej się w~ramy współczesnych systemów Internetu Rzeczy.
    


\end{pl_abstract}

\begin{slowa_klucze}
    czujnik ultradźwiękowy, zbiornik asenizacyjny, Internet Rzeczy, interfejs LoRa, STM32WL, monitoring, komunikacja bezprzewodowa,
    pomiar bezkontaktowy, poziom cieczy.

\end{slowa_klucze}

\clearpage
\begin{eng_abstract}
Nowadays, there is a~lack of economical, wireless, and contactless solutions for monitoring the fill levels of domestic septic tanks.
This thesis presents the design of a~system addressing these needs. The developed solution consists of two separate devices: a~measurement module
employing an ultrasonic sensor for liquid level detection, and a~base station — both built on STM32WL microcontrollers with an integrated LoRa
radio interface. The conducted tests confirmed the correct operation of the modules, which lay a~foundation for the development of a~comprehensive
monitoring platform fitting into the framework of modern Internet of Things systems.
\end{eng_abstract}


\begin{second_keywords}
    ultrasonic sensor, septic tank, Internet of Things, LoRa interface, STM32WL, monitoring, wireless communication, contactless measurements, fill level.
\end{second_keywords}